\documentclass[a4paper,12pt]{article}
\usepackage[utf8]{inputenc}
\usepackage[ukrainian]{babel}
\usepackage{amsmath}
\usepackage{graphicx}
\usepackage{geometry}
\usepackage{float}
\usepackage{listings}
\usepackage{color}
\usepackage{hyperref}

\geometry{
 a4paper,
 total={170mm,257mm},
 left=20mm,
 top=20mm,
}

\begin{document}

% Титульна сторінка
\begin{titlepage}
\centering

\vspace*{1cm}

{\large \textbf{Київський національний університет імені Тараса Шевченка}}

\vspace{0.3cm}

{\large Факультет комп'ютерних наук та кібернетики}

\vspace{3cm}

{\Large \textbf{ЗВІТ}}

\vspace{0.5cm}

{\large з виконання завдання із некласичної оптимізації}

\vspace{1cm}

{\Large \textbf{Тема:}}

{\Large Диференціальна еволюція для параметричної оптимізації}

\vspace{0.5cm}

{\large Завдання 6}

\vfill

\begin{flushright}
{\large Виконав:}\\
{\large Кіщук Ярослав Ярославович}

\vspace{1cm}

{\large Викладач:}\\
{\large Джалладова Ірада Агаєвна}
\end{flushright}

\vfill

{\large Київ — 2025}

\end{titlepage}

\newpage
\tableofcontents
\newpage

\section{Постановка задачі}

Метою роботи є застосування методу Диференціальної Еволюції (Differential Evolution, DE) для калібрування параметрів нелінійної регресійної моделі. Необхідно ідентифікувати 8 невідомих параметрів кінетичної моделі, що описує залежність швидкості хімічної реакції від температури та концентрації реагентів.

Основні етапи роботи:
\begin{enumerate}
    \item Реалізація кінетичної моделі.
    \item Генерація синтетичних експериментальних даних із накладанням шуму.
    \item Побудова цільової функції (середньоквадратичне відхилення).
    \item Застосування алгоритму DE з трьома різними стратегіями мутації.
    \item Статистичний аналіз результатів за 50 незалежними запусками.
\end{enumerate}

\section{Математична модель}

У роботі розглядається \textbf{узагальнена кінетична модель} швидкості реакції $r(T, C_A, C_B; \boldsymbol{\theta})$, яка поєднує типові елементи хімічної кінетики:
експоненційну залежність від температури за законом Арреніуса, степеневі залежності від концентрацій реагентів та додатковий множник, що може описувати ефекти інгібування/активації. Модель не запозичена з конкретної публікації, а побудована як \textbf{синтетичний навчальний приклад} на основі типових формул хімічної кінетики для демонстрації роботи алгоритмів глобальної оптимізації.

Безрозмірна форма моделі має вигляд:

\begin{equation}
r = \theta_1 T^{\theta_2} \exp\left( -\frac{\theta_3}{RT} \right) C_A^{\theta_4} C_B^{\theta_5} \left(1 + \theta_6 C_A + \theta_7 C_B \right)^{\theta_8}
\end{equation}

де:
\begin{itemize}
    \item $T$ — температура (K);
    \item $C_A, C_B$ — концентрації реагентів;
    \item $R = 8.314$ Дж/(моль$\cdot$K) — газова стала;
    \item $\boldsymbol{\theta} = (\theta_1, \dots, \theta_8)$ — вектор невідомих параметрів.
\end{itemize}

Цільова функція $J(\boldsymbol{\theta})$ визначається як середньоквадратична помилка (MSE) між модельними значеннями та експериментальними даними:

\begin{equation}
J(\boldsymbol{\theta}) = \frac{1}{N} \sum_{k=1}^{N} \left( r_{\text{model}}^{(k)} - r_{\text{exp}}^{(k)} \right)^2
\end{equation}

\section{Алгоритм Диференціальної Еволюції}

\subsection{Загальний опис методу}

Differential Evolution (DE) — це популяційний еволюційний алгоритм глобальної оптимізації для неперервних задач, запропонований Storn та Price у 1997 році. Алгоритм оперує популяцією з $NP$ векторів-кандидатів у $D$-вимірному просторі:

\[
\mathbf{x}_i^{(g)} = (x_{i,1}^{(g)}, \dots, x_{i,D}^{(g)}), \quad i = 1,\dots,NP
\]

де $g$ — номер покоління.

\subsection{Основні операції алгоритму}

Кожна ітерація DE складається з трьох основних етапів:

\subsubsection{Мутація}

Для кожного цільового вектора $\mathbf{x}_i$ створюється донорний вектор $\mathbf{v}_i$ за однією з стратегій:

\begin{itemize}
    \item \textbf{DE/rand/1}: $\mathbf{v}_i = \mathbf{x}_{r1} + F(\mathbf{x}_{r2} - \mathbf{x}_{r3})$
    
    Базова стратегія з випадковим вибором базового вектора. Забезпечує високу різноманітність популяції.
    
    \item \textbf{DE/best/1}: $\mathbf{v}_i = \mathbf{x}_{\text{best}} + F(\mathbf{x}_{r1} - \mathbf{x}_{r2})$
    
    Використовує найкращий вектор як базовий. Забезпечує швидку збіжність, але може призводити до передчасного збігання.
    
    \item \textbf{DE/current-to-best/1}: $\mathbf{v}_i = \mathbf{x}_i + F(\mathbf{x}_{\text{best}} - \mathbf{x}_i) + F(\mathbf{x}_{r1} - \mathbf{x}_{r2})$
    
    Комбінована стратегія, що балансує між експлорацією та експлуатацією.
\end{itemize}

Тут $F \in (0, 2)$ — коефіцієнт масштабування (у нашому випадку $F = 0.8$), а $r1, r2, r3$ — випадкові індекси, різні між собою та відмінні від $i$.

\subsubsection{Кросовер}

Формується випробувальний вектор $\mathbf{u}_i$ шляхом біноміального схрещування:

\[
u_{i,j} = \begin{cases}
v_{i,j}, & \text{якщо } \text{rand}_j \leq CR \text{ або } j = j_{\text{rand}} \\
x_{i,j}, & \text{інакше}
\end{cases}
\]

де $CR \in [0,1]$ — ймовірність схрещування (у нашому випадку $CR = 0.9$), $j_{\text{rand}}$ — випадково обраний індекс, що гарантує зміну хоча б однієї координати.

\subsubsection{Селекція}

Застосовується жадібна селекція:

\[
\mathbf{x}_i^{(g+1)} = \begin{cases}
\mathbf{u}_i, & \text{якщо } f(\mathbf{u}_i) \leq f(\mathbf{x}_i^{(g)}) \\
\mathbf{x}_i^{(g)}, & \text{інакше}
\end{cases}
\]

Це забезпечує монотонне покращення якості популяції.

\subsection{Налаштування параметрів}

Для даної задачі використано наступні параметри:
\begin{itemize}
    \item Розмір популяції: $NP = 48 \approx 50$ (близько до рекомендованого $5D$ до $10D$)
    \item Коефіцієнт масштабування: $F = 0.8$
    \item Ймовірність схрещування: $CR = 0.9$
    \item Критерій зупинки: 200 поколінь
\end{itemize}

\section{Програмна реалізація}

\subsection{Використані бібліотеки}

Реалізацію виконано мовою Python 3.x з використанням наступних бібліотек:

\subsubsection{NumPy}

Бібліотека для ефективних векторних та матричних обчислень. Використовувалась для:
\begin{itemize}
    \item Генерації сіток даних через \texttt{np.meshgrid}
    \item Векторизованих обчислень кінетичної моделі
    \item Операцій з масивами параметрів
\end{itemize}

\subsubsection{SciPy}

Функція \texttt{scipy.optimize.differential\_evolution} надає готову реалізацію DE з підтримкою різних стратегій мутації. Основні параметри:

\begin{verbatim}
differential_evolution(
    func,           # цільова функція
    bounds,         # межі параметрів
    strategy,       # стратегія мутації
    maxiter,        # максимальна кількість поколінь
    popsize,        # множник розміру популяції
    mutation,       # коефіцієнт F
    recombination,  # ймовірність CR
    seed,           # seed для відтворюваності
    polish=False    # вимкнути локальну оптимізацію
)
\end{verbatim}

У SciPy розмір популяції визначається як $NP = \texttt{popsize} \times D$. Для отримання $NP = 48$ при $D = 8$ використано \texttt{popsize = 6}.

\subsubsection{Pymoo/Pymoode}

Бібліотека \texttt{pymoo} — потужний фреймворк для багатокритеріальної оптимізації з підтримкою різних еволюційних алгоритмів. Для DE використано модуль \texttt{pymoode}, що надає розширену реалізацію з підтримкою стратегії \texttt{DE/current-to-best/1/bin}.

Реалізація через клас задачі:

\begin{verbatim}
class KineticProblem(ElementwiseProblem):
    def __init__(self):
        super().__init__(
            n_var=8,
            n_obj=1,
            n_constr=0,
            xl=lower_bounds,
            xu=upper_bounds
        )
    
    def _evaluate(self, x, out, *args, **kwargs):
        out["F"] = objective_function(x)
\end{verbatim}

Запуск оптимізації:

\begin{verbatim}
algorithm = DE(
    pop_size=48,
    variant="DE/current-to-best/1/bin",
    F=0.8,
    CR=0.9
)

res = minimize(
    problem,
    algorithm,
    termination,
    seed=seed,
    verbose=False
)
\end{verbatim}

\subsection{Структура коду}

Програма організована в наступні основні компоненти:

\begin{enumerate}
    \item \textbf{Модель}: Функція \texttt{rate\_model()} обчислює швидкість реакції для заданих параметрів
    \item \textbf{Цільова функція}: Функція \texttt{objective()} обчислює MSE між модельними та експериментальними даними
    \item \textbf{Обгортки DE}: Функції для запуску різних варіантів DE з уніфікованим інтерфейсом
    \item \textbf{Цикл експериментів}: 50 незалежних запусків для кожної стратегії з різними seed
    \item \textbf{Аналіз результатів}: Збір статистики, побудова графіків, порівняння параметрів
\end{enumerate}

\section{Результати експериментів}

Було проведено 50 незалежних запусків для кожної стратегії. Загалом використано 1000 експериментальних точок, згенерованих з істинної моделі з накладанням 5\% шуму.

\subsection{Візуалізація кінетичної моделі}

Для кращого розуміння складності оптимізаційної задачі, на рис.~\ref{fig:surface3d} показано тривимірну поверхню кінетичної функції $r(T, C_A)$ при фіксованому значенні $C_B = 1.0$. Графік демонструє сильну нелінійність моделі та експоненціальне зростання швидкості реакції зі збільшенням температури.

\begin{figure}[H]
\centering
\includegraphics[width=0.8\textwidth]{surface_3d.png}
\caption{3D поверхня кінетичної функції $r(T, C_A)$ при $C_B = 1.0$}
\label{fig:surface3d}
\end{figure}

\subsection{Синтетичні дані}

Для тестування алгоритму DE використано метод синтетичних (штучних) даних. Цей підхід дозволяє контролювати істинні значення параметрів та оцінити якість їх відновлення.

Процес генерації даних:
\begin{enumerate}
    \item Задано вектор \textbf{істинних параметрів} $\boldsymbol{\theta}^{\text{true}} = (\theta_1 = 5.0, \theta_2 = 0.5, \dots, \theta_8 = 2.0)$.
    \item Для сітки значень $T \in [300, 500]$ K, $C_A \in [0.5, 1.5]$, $C_B \in [0.4, 1.2]$ обчислено \textbf{істинні значення} швидкості реакції:
    \[
    r_k^{\text{true}} = r(T_k, C_{A,k}, C_{B,k}; \boldsymbol{\theta}^{\text{true}})
    \]
    \item До істинних значень додано \textbf{гаусівський шум} з рівнем 5\% для імітації експериментальних похибок:
    \[
    r_k^{\text{exp}} = r_k^{\text{true}} \cdot (1 + \varepsilon_k), \quad \varepsilon_k \sim \mathcal{N}(0, 0.05^2)
    \]
\end{enumerate}

Такий підхід моделює реальну ситуацію, коли дослідник має лише зашумлені експериментальні виміри та не знає точних значень параметрів моделі. Мета алгоритму оптимізації — відновити параметри $\boldsymbol{\theta}$, використовуючи тільки експериментальні дані $r_k^{\text{exp}}$.

На рис.~\ref{fig:data} показано згенеровані експериментальні дані у порівнянні з істинними значеннями моделі. Помаранчеві точки представляють ідеальні значення без шуму, тоді як сині точки — експериментальні дані з накладеним випадковим шумом. Видно, що дані містять розкид навколоістинних значень, що ускладнює задачу ідентифікації параметрів та перевіряє робастність алгоритму DE.

\begin{figure}[H]
\centering
\includegraphics[width=0.7\textwidth]{data_plot.png}
\caption{Порівняння експериментальних даних з істинними значеннями моделі. Помаранчеві точки — істинні значення, сині — експериментальні дані з 5\% шумом.}
\label{fig:data}
\end{figure}

\subsection{3D візуалізація кінетичної моделі}

Для кращого розуміння поведінки кінетичної моделі створено серію 3D візуалізацій з фіксуванням різних параметрів. Ці графіки дозволяють проаналізувати залежність швидкості реакції від різних комбінацій вхідних змінних.

На рис.~\ref{fig:surface_tcb} показано залежність швидкості реакції від температури T та концентрації реагенту B ($C_B$) при фіксованій концентрації $C_A = 1.0$. Спостерігається експоненціальне зростання швидкості реакції з підвищенням температури, що відповідає закону Арреніуса.

\begin{figure}[H]
\centering
\includegraphics[width=0.7\textwidth]{surface_T_CB.png}
\caption{3D поверхня $r(T, C_B)$ при фіксованому $C_A = 1.0$}
\label{fig:surface_tcb}
\end{figure}

На рис.~\ref{fig:surface_cacb} представлено залежність від концентрацій обох реагентів при фіксованій температурі T = 400 K. Ця візуалізація демонструє взаємний вплив концентрацій на швидкість реакції.

\begin{figure}[H]
\centering
\includegraphics[width=0.7\textwidth]{surface_CA_CB.png}
\caption{3D поверхня $r(C_A, C_B)$ при фіксованій температурі T = 400 K}
\label{fig:surface_cacb}
\end{figure}

На рис.~\ref{fig:surface_combined} показано комбіновану візуалізацію з чотирма підграфіками, що ілюструють поведінку моделі при різних фіксованих параметрах. Це дозволяє порівняти вплив кожного параметра на форму поверхні відгуку.

\begin{figure}[H]
\centering
\includegraphics[width=1.0\textwidth]{surface_combined.png}
\caption{Комбінована візуалізація кінетичної моделі: (верхній ряд) $r(T, C_A)$ при $C_B = 0.5$ та $C_B = 1.5$; (нижній ряд) $r(T, C_B)$ при $C_A = 0.7$ та $r(C_A, C_B)$ при T = 450 K}
\label{fig:surface_combined}
\end{figure}

\subsection{Точність та збіжність}

Порівняння середнього значення функції помилки (MSE) та стандартного відхилення:

\begin{table}[H]
\centering
\begin{tabular}{|l|c|c|}
\hline
\textbf{Стратегія} & \textbf{Mean MSE} & \textbf{Std Dev} \\
\hline
DE/rand/1/bin & $9.146 \times 10^{-17}$ & $2.781 \times 10^{-17}$ \\
\hline
DE/best/1/bin & $5.400 \times 10^{-17}$ & $1.421 \times 10^{-18}$ \\
\hline
DE/current-to-best/1 & $5.421 \times 10^{-17}$ & $2.215 \times 10^{-18}$ \\
\hline
\end{tabular}
\caption{Статистика помилки за 50 запусків}
\end{table}

На рис.~\ref{fig:boxplot} наведено діаграму розмаху (boxplot) для порівняння розподілів MSE. Видно, що стратегії \textbf{DE/best/1} та \textbf{DE/current-to-best/1} показують кращу стабільність і нижчі значення помилки порівняно з \textbf{DE/rand/1}.

\begin{figure}[H]
\centering
\includegraphics[width=0.75\textwidth]{boxplot.png}
\caption{Порівняння розподілу MSE для трьох стратегій}
\label{fig:boxplot}
\end{figure}

\subsection{Швидкість збіжності}

На рис.~\ref{fig:convergence} показано динаміку збіжності для одного характерного запуску кожної стратегії. Графік демонструє, що стратегії \textbf{DE/best/1} та \textbf{DE/current-to-best/1} досягають оптимуму швидше, ніж \textbf{DE/rand/1}.

\begin{figure}[H]
\centering
\includegraphics[width=0.75\textwidth]{convergence.png}
\caption{Збіжність алгоритмів по поколіннях}
\label{fig:convergence}
\end{figure}

\subsection{Відновлення параметрів}

Порівняння знайдених параметрів з істинними значеннями для найкращого запуску кожної стратегії:

\begin{table}[H]
\centering
\begin{tabular}{|c|c|c|c|c|}
\hline
Параметр & Істинне & DE/rand/1 & DE/best/1 & DE/curr-to-best \\
\hline
$\theta_1$ & 5.0 & 841.26 & 687.28 & 429.06 \\
$\theta_2$ & 0.5 & $-0.250$ & $-0.115$ & $-0.060$ \\
$\theta_3$ & 80000.0 & 82276.5 & 83651.4 & 83214.9 \\
$\theta_4$ & 1.0 & 1.043 & 1.120 & 1.161 \\
$\theta_5$ & 1.2 & 1.095 & 1.245 & 1.192 \\
$\theta_6$ & 0.1 & 1.253 & 0.015 & 0.021 \\
$\theta_7$ & $-0.05$ & $-0.047$ & $-0.044$ & $-0.405$ \\
$\theta_8$ & 2.0 & 0.231 & 3.728 & 0.166 \\
\hline
\end{tabular}
\caption{Порівняння параметрів}
\end{table}

Найкращі значення MSE для кожної стратегії:
\begin{itemize}
    \item DE/rand/1: $5.675 \times 10^{-17}$
    \item DE/best/1: $5.290 \times 10^{-17}$
    \item DE/current-to-best/1: $5.296 \times 10^{-17}$
\end{itemize}

\section{Висновки}

В ході роботи було успішно реалізовано та протестовано алгоритм Диференціальної Еволюції для задачі ідентифікації параметрів кінетичної моделі з 8 невідомими параметрами.

Основні результати дослідження:

\begin{enumerate}
    \item \textbf{Точність оптимізації:} Всі три стратегії DE досягли дуже низьких значень MSE (порядку $10^{-17}$), що свідчить про успішну мінімізацію цільової функції.
    
    \item \textbf{Порівняння стратегій:}
    \begin{itemize}
        \item \textbf{DE/best/1} показала найкращі результати за середнім значенням MSE ($5.400 \times 10^{-17}$) та найменшу дисперсію результатів, що свідчить про високу надійність цієї стратегії.
        \item \textbf{DE/current-to-best/1} продемонструвала результати, порівнянні з DE/best/1, з MSE $5.421 \times 10^{-17}$ та дещо більшою варіативністю.
        \item \textbf{DE/rand/1} показала найгірші результати (MSE $9.146 \times 10^{-17}$) з найбільшою дисперсією, що підтверджує більшу стохастичність базової стратегії.
    \end{itemize}
    
    \item \textbf{Швидкість збіжності:} Графіки збіжності показують, що стратегії на основі найкращого індивіда (DE/best/1 та DE/current-to-best/1) досягають оптимуму швидше, ніж випадкова стратегія DE/rand/1.
    
    \item \textbf{Відновлення параметрів:} Незважаючи на дуже низьке значення MSE, деякі параметри моделі були відновлені з великими відхиленнями від істинних значень (наприклад, $\theta_1$, $\theta_2$, $\theta_8$). Це вказує на наявність сильної кореляції між параметрами та існування множини локально-еквівалентних рішень, які дають практично однакові прогнози моделі при різних комбінаціях параметрів.
\end{enumerate}

Практичні висновки:
\begin{itemize}
    \item Для задач параметричної оптимізації рекомендується використовувати стратегії \textbf{DE/best/1} або \textbf{DE/current-to-best/1} через їх кращу збіжність та стабільність.
    \item Низьке значення цільової функції не завжди гарантує точне відновлення параметрів моделі у присутності кореляцій між параметрами.
    \item Налаштування гіперпараметрів ($NP=48$, $F=0.8$, $CR=0.9$) виявилися ефективними для даної задачі розмірності $D=8$.
\end{itemize}

\end{document}
