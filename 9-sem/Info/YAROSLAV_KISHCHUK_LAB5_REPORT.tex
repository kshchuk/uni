\documentclass[a4paper,12pt]{article}
\usepackage[T2A]{fontenc}
\usepackage[utf8]{inputenc}
\usepackage[ukrainian]{babel}
\usepackage{geometry}
\geometry{margin=20mm}
\usepackage{hyperref}
\usepackage{longtable}
\usepackage{booktabs}
\usepackage{array}
\usepackage{listings}
\usepackage{xcolor}
\usepackage{graphicx}

\lstdefinestyle{sqlstyle}{%
  language=SQL,
  basicstyle=\ttfamily\small,
  keywordstyle=\color{blue}\bfseries,
  stringstyle=\color{red!60!black},
  commentstyle=\color{green!50!black}\itshape,
  showstringspaces=false,
  frame=single,
  breaklines=true,
  tabsize=2,
  numbers=left,
  numberstyle=\tiny\color{gray}
}

\lstdefinestyle{pythonstyle}{%
  language=Python,
  basicstyle=\ttfamily\small,
  keywordstyle=\color{blue}\bfseries,
  stringstyle=\color{red!60!black},
  commentstyle=\color{green!50!black}\itshape,
  showstringspaces=false,
  frame=single,
  breaklines=true,
  tabsize=2
}

\begin{document}

\begin{titlepage}
  \begin{center}
    \Huge \textbf{Лабораторна робота №5}\\[4mm]
    \LARGE Реалізація бізнес-аналітики для системи TechMarket\\[3cm]
    \large Виконав: Кіщук Ярослав\\[2mm]
  \end{center}
\end{titlepage}

\tableofcontents
\newpage

\section{Вступ}
Метою п'ятої лабораторної роботи є розробка системи бізнес-аналітики (BI) для інтернет-магазину TechMarket. У попередніх лабораторних роботах було спроєктовано архітектуру баз даних (OLTP та DWH), реалізовано ETL-процес для завантаження даних із операційних баз у аналітичне сховище. Тепер завданням є створення аналітичних запитів для розрахунку ключових показників ефективності (KPI) та побудова інтерактивних дашбордів для візуалізації цих метрик.

У рамках роботи використано:
\begin{itemize}
  \item \textbf{PostgreSQL DWH} --- аналітичне сховище даних із зоряною схемою (6 dimension-таблиць та 1 fact-таблиця).
  \item \textbf{SQL} --- мова запитів для розрахунку KPI безпосередньо у базі даних.
  \item \textbf{Metabase} --- open-source BI-платформа для візуалізації даних та створення дашбордів.
  \item \textbf{Python} --- автоматизація налаштування Metabase через API (\texttt{scripts/metabase\_seed.py}).
\end{itemize}

\section{Мета та завдання}
{\renewcommand{\labelenumi}{\textbf{Завдання \theenumi:}}
\begin{enumerate}
  \item Визначити п'ять ключових показників ефективності (KPI) для аналітики продажів TechMarket.
  \item Розробити SQL-запити для розрахунку кожного KPI на основі зоряної схеми DWH.
  \item Реалізувати параметризовані запити з фільтрацією за датами та регіонами.
  \item Інтегрувати Metabase з PostgreSQL DWH для візуалізації даних.
  \item Створити інтерактивний дашборд із п'ятьма візуалізаціями та глобальними фільтрами.
  \item Автоматизувати розгортання BI-інфраструктури через Python-скрипт.
\end{enumerate}}

\section{Архітектура аналітичного рішення}
\subsection{Структура DWH}
Аналітичне сховище даних побудовано за принципом зоряної схеми (star schema), що забезпечує оптимальну продуктивність аналітичних запитів. Центральним елементом є таблиця фактів \texttt{fact\_sales}, яка містить метрики продажів та зовнішні ключі до вимірів.

\begin{longtable}{>{\raggedright\arraybackslash}p{0.25\textwidth} p{0.70\textwidth}}
\caption{Структура DWH схеми}\label{tab:dwh}\\
\toprule
Таблиця & Опис \\
\midrule
\texttt{fact\_sales} & Факт продажів: \texttt{order\_id}, \texttt{date\_key}, \texttt{product\_key}, \texttt{customer\_key}, \texttt{employee\_key}, \texttt{region\_key}, \texttt{quantity}, \texttt{revenue}, \texttt{discount\_amount}, \texttt{cost}, \texttt{margin} \\
\texttt{dim\_date} & Календар: \texttt{date\_key} (YYYYMMDD), \texttt{date}, \texttt{year}, \texttt{quarter}, \texttt{month}, \texttt{day}, \texttt{day\_of\_week}, \texttt{is\_weekend} \\
\texttt{dim\_product} & Товари: \texttt{product\_key}, \texttt{product\_id}, \texttt{name}, \texttt{sku}, \texttt{category\_key} \\
\texttt{dim\_customer} & Клієнти: \texttt{customer\_key}, \texttt{customer\_id}, \texttt{first\_name}, \texttt{last\_name}, \texttt{email}, \texttt{region\_key} \\
\texttt{dim\_employee} & Менеджери: \texttt{employee\_key}, \texttt{employee\_id}, \texttt{first\_name}, \texttt{last\_name}, \texttt{email}, \texttt{region\_key} \\
\texttt{dim\_region} & Регіони: \texttt{region\_key}, \texttt{region\_id}, \texttt{name}, \texttt{code} \\
\texttt{dim\_category} & Категорії: \texttt{category\_key}, \texttt{category\_id}, \texttt{name}, \texttt{parent\_category\_key} \\
\bottomrule
\end{longtable}

\subsection{Інтеграція з Metabase}
Metabase розгорнуто як Docker-контейнер (\texttt{metabase/metabase:v0.48.6}) і підключено до PostgreSQL DWH через нативний драйвер. Конфігурація здійснюється автоматично за допомогою Python-скрипта \texttt{metabase\_seed.py}, який виконує:
\begin{itemize}
  \item Автентифікацію в Metabase API
  \item Створення підключення до DWH
  \item Генерацію SQL-запитів із template tags для параметризації
  \item Створення Saved Questions (збережених запитів) для кожного KPI
  \item Побудову дашборда з розміщенням візуалізацій та налаштуванням фільтрів
\end{itemize}

\section{Ключові показники ефективності (KPI)}
Для аналізу ефективності продажів системи TechMarket визначено п'ять KPI, які покривають різні аспекти бізнесу: тренд виручки, географічний розподіл замовлень, середній чек, маржинальність та популярність товарів.

\subsection{KPI 1: Динаміка виручки за місяцями (Revenue by Month)}
\textbf{Бізнес-питання:} Як змінюється виручка компанії в часі? Які місяці є найбільш прибутковими?

\textbf{Опис:} Цей показник відображає сумарну виручку, знижки та маржу за кожен місяць. Дозволяє виявити сезонність, тренди зростання або падіння продажів.

\textbf{SQL-запит:}
\begin{lstlisting}[style=sqlstyle, caption={Динаміка виручки за місяцями}]
SELECT
  d.year,
  d.month,
  to_char(d.date, 'Mon') AS month_name,
  SUM(f.revenue)          AS revenue,
  SUM(f.discount_amount)  AS discount,
  SUM(f.margin)           AS margin
FROM fact_sales f
JOIN dim_date d ON f.date_key = d.date_key
LEFT JOIN dim_region r ON f.region_key = r.region_key
WHERE 1=1
  AND d.date >= '2024-01-01'
  AND d.date <= '2024-12-31'
  AND r.name = 'Kyivska'
GROUP BY d.year, d.month, to_char(d.date, 'Mon')
ORDER BY d.year, d.month;
\end{lstlisting}

\textbf{Тип візуалізації:} Line chart (лінійний графік) для відображення трендів у часі.

\subsection{KPI 2: Кількість замовлень за регіонами (Orders by Region)}
\textbf{Бізнес-питання:} Які регіони генерують найбільше замовлень? Яка географічна структура продажів?

\textbf{Опис:} Показник агрегує кількість унікальних замовлень та сумарну виручку в розрізі регіонів. Допомагає виявити найбільш активні ринки збуту.

\textbf{SQL-запит:}
\begin{lstlisting}[style=sqlstyle, caption={Замовлення за регіонами}]
SELECT
  r.name AS region_name,
  COUNT(DISTINCT f.order_id) AS orders_count,
  SUM(f.revenue)             AS revenue
FROM fact_sales f
JOIN dim_region r ON f.region_key = r.region_key
WHERE 1=1
  AND f.date_key >= 20240101
  AND f.date_key <= 20241231
GROUP BY r.name
ORDER BY orders_count DESC;
\end{lstlisting}

\textbf{Тип візуалізації:} Bar chart (стовпчикова діаграма) для порівняння регіонів.

\subsection{KPI 3: Середній чек (Average Order Value)}
\textbf{Бізнес-питання:} Скільки в середньому витрачає клієнт за одне замовлення?

\textbf{Опис:} Метрика розраховується як відношення сумарної виручки до кількості унікальних замовлень. Є індикатором якості продажів та ефективності маркетингових акцій.

\textbf{SQL-запит:}
\begin{lstlisting}[style=sqlstyle, caption={Середній чек}]
SELECT
  SUM(f.revenue) / NULLIF(COUNT(DISTINCT f.order_id), 0) 
    AS avg_order_value
FROM fact_sales f
LEFT JOIN dim_region r ON f.region_key = r.region_key
JOIN dim_date d ON f.date_key = d.date_key
WHERE 1=1
  AND d.date >= '2024-01-01'
  AND d.date <= '2024-12-31'
  AND r.name = 'Kyivska';
\end{lstlisting}

\textbf{Тип візуалізації:} Scalar (число) --- одне велике значення для швидкого огляду.

\subsection{KPI 4: Відсоток маржі (Margin Percentage)}
\textbf{Бізнес-питання:} Яка частка прибутку в загальній виручці?

\textbf{Опис:} Показник відображає відношення маржі (різниці між виручкою та собівартістю) до виручки у відсотках. Критичний для оцінки рентабельності бізнесу.

\textbf{SQL-запит:}
\begin{lstlisting}[style=sqlstyle, caption={Відсоток маржі}]
SELECT
  (SUM(f.margin) / NULLIF(SUM(f.revenue), 0))::numeric(12,4) 
    AS margin_pct
FROM fact_sales f
LEFT JOIN dim_region r ON f.region_key = r.region_key
JOIN dim_date d ON f.date_key = d.date_key
WHERE 1=1
  AND d.date >= '2024-01-01'
  AND d.date <= '2024-12-31'
  AND r.name = 'Kyivska';
\end{lstlisting}

\textbf{Тип візуалізації:} Scalar з форматуванням у відсотки.

\subsection{KPI 5: Топ-10 товарів за виручкою (Top Products by Revenue)}
\textbf{Бізнес-питання:} Які товари приносять найбільше доходу? На які товари варто робити акцент у маркетингу?

\textbf{Опис:} Рейтинг десяти найприбутковіших товарів із відображенням виручки, проданої кількості та маржі.

\textbf{SQL-запит:}
\begin{lstlisting}[style=sqlstyle, caption={Топ-10 товарів}]
SELECT
  p.name AS product_name,
  SUM(f.revenue) AS revenue,
  SUM(f.quantity) AS qty,
  SUM(f.margin) AS margin
FROM fact_sales f
JOIN dim_product p ON f.product_key = p.product_key
LEFT JOIN dim_region r ON f.region_key = r.region_key
JOIN dim_date d ON f.date_key = d.date_key
WHERE 1=1
  AND d.date >= '2024-01-01'
  AND d.date <= '2024-12-31'
  AND r.name = 'Kyivska'
GROUP BY p.name
ORDER BY revenue DESC
LIMIT 10;
\end{lstlisting}

\textbf{Тип візуалізації:} Bar chart (горизонтальні стовпці) для зручного порівняння товарів.

\section{Параметризація запитів}
Усі п'ять KPI підтримують динамічну фільтрацію через Metabase template tags:
\begin{itemize}
  \item \texttt{\{\{start\_date\}\}} --- початкова дата періоду аналізу (тип: \texttt{date})
  \item \texttt{\{\{end\_date\}\}} --- кінцева дата періоду аналізу (тип: \texttt{date})
  \item \texttt{\{\{region\}\}} --- назва регіону для фільтрації (тип: \texttt{text}, опціональний)
\end{itemize}

Синтаксис Metabase для опціональних параметрів: \texttt{[[AND condition]]}, що дозволяє виключити фільтр, якщо параметр не заданий.

\section{Створення дашборда в Metabase}
\subsection{Архітектура дашборда}
Дашборд ``TechMarket KPI Dashboard'' складається з п'яти візуалізацій, організованих у grid-layout:
\begin{itemize}
  \item \textbf{Row 0--6:} Revenue by Month (ширина 12 колонок) --- займає всю ширину екрана
  \item \textbf{Row 6--12:} Orders by Region (6 колонок), Average Order Value (3 колонки), Margin Percentage (3 колонки)
  \item \textbf{Row 12--18:} Top Products by Revenue (12 колонок)
\end{itemize}

\subsection{Глобальні фільтри}
На рівні дашборда налаштовано три параметри:
\begin{enumerate}
  \item \textbf{Start Date} --- вибір початкової дати (date picker)
  \item \textbf{End Date} --- вибір кінцевої дати (date picker)
  \item \textbf{Region} --- текстове поле для назви регіону (опціонально)
\end{enumerate}

Кожна візуалізація автоматично прив'язана до цих параметрів через \texttt{parameter\_mappings}, що забезпечує синхронну фільтрацію всіх графіків.

\subsection{Автоматизація розгортання}
Для автоматичного налаштування Metabase створено Python-скрипт \texttt{metabase\_seed.py}, який:
\begin{enumerate}
  \item Автентифікується в Metabase API за допомогою \texttt{/api/session}
  \item Створює або знаходить існуюче підключення до PostgreSQL DWH
  \item Генерує п'ять Saved Questions з відповідними SQL-запитами та template tags
  \item Створює дашборд та налаштовує глобальні параметри
  \item Додає всі візуалізації на дашборд із відповідним layout та mappings
\end{enumerate}

\textbf{Приклад запуску:}
\begin{lstlisting}[style=pythonstyle]
# Setup environment variables
export METABASE_URL=http://localhost:3000
export METABASE_USER=admin@example.com
export METABASE_PASS=secret

# Run the seed script
python scripts/metabase_seed.py
\end{lstlisting}

Скрипт також підтримує прапорець \texttt{--clear} для видалення попередніх версій дашборда та запитів перед створенням нових.

\section{Інтеграція з інфраструктурою}
\subsection{Docker Compose}
Metabase інтегровано в загальну інфраструктуру TechMarket через \texttt{docker-compose.yml}:
\begin{lstlisting}[style=pythonstyle]
services:
  metabase:
    image: metabase/metabase:v0.48.6
    container_name: techmarket-metabase
    ports:
      - "3000:3000"
    environment:
      MB_DB_FILE: /metabase-data/metabase.db
    volumes:
      - metabase-data:/metabase-data
    networks:
      - techmarket-net
    depends_on:
      - dwh-db
\end{lstlisting}

\subsection{Послідовність запуску}
\begin{enumerate}
  \item Запуск всієї інфраструктури: \texttt{docker-compose up -d}
  \item Генерація тестових даних в OLTP: \texttt{python database/data/generate\_test\_data.py}
  \item Виконання ETL: \texttt{python etl/run\_etl.py --mode full}
  \item Налаштування Metabase: \texttt{python scripts/metabase\_seed.py}
  \item Відкриття дашборда: \texttt{http://localhost:3000}
\end{enumerate}

\section{Результати та аналіз}
\subsection{Досягнуті результати}
У рамках лабораторної роботи успішно реалізовано:
\begin{itemize}
  \item 5 ключових показників ефективності (KPI) для аналізу продажів TechMarket
  \item SQL-запити з підтримкою параметризації та фільтрації за датами/регіонами
  \item Інтерактивний дашборд у Metabase із п'ятьма візуалізаціями
  \item Автоматизацію розгортання BI-інфраструктури через Python API
  \item Інтеграцію Metabase з PostgreSQL DWH та Docker-екосистемою
\end{itemize}

\subsection{Аналіз KPI}
Розроблені показники дозволяють відповісти на ключові бізнес-питання:
\begin{enumerate}
  \item \textbf{Revenue by Month} --- виявлення трендів та сезонності продажів
  \item \textbf{Orders by Region} --- визначення найбільш активних географічних ринків
  \item \textbf{Average Order Value} --- оцінка ефективності стратегій підвищення чека
  \item \textbf{Margin Percentage} --- контроль рентабельності бізнесу
  \item \textbf{Top Products} --- фокусування маркетингових зусиль на найприбутковіших товарах
\end{enumerate}

\subsection{Переваги використання Metabase}
\begin{itemize}
  \item \textbf{Open-source:} безкоштовна альтернатива Power BI та Tableau
  \item \textbf{Простота інтеграції:} нативна підтримка PostgreSQL без додаткових драйверів
  \item \textbf{REST API:} повна автоматизація через програмний інтерфейс
  \item \textbf{Інтерактивність:} фільтри, drill-down, експорт у CSV/PDF
  \item \textbf{Легке розгортання:} один Docker-контейнер без складних налаштувань
\end{itemize}

\section{Висновки}
У ході виконання лабораторної роботи №5 було успішно реалізовано повноцінну систему бізнес-аналітики для інтернет-магазину TechMarket. Розроблено п'ять ключових показників ефективності, які покривають різні аспекти діяльності компанії: динаміку виручки, географічний розподіл продажів, середній чек, маржинальність та популярність товарів.

Для кожного KPI створено оптимізовані SQL-запити, які використовують зоряну схему DWH та забезпечують швидке виконання аналітичних операцій. Параметризація запитів дозволяє динамічно фільтрувати дані за датами та регіонами, що підвищує гнучкість аналізу.

Інтеграція з Metabase забезпечила створення інтерактивного дашборда з професійними візуалізаціями. Автоматизація налаштування через Python API дозволяє швидко відтворювати BI-інфраструктуру в різних середовищах (development, staging, production).

Обране рішення (Metabase) виявилося оптимальним для навчальних та малих комерційних проектів завдяки простоті використання, відсутності ліцензійних витрат та повній інтеграції з існуючою Docker-інфраструктурою проекту.

Результати роботи можуть бути використані для прийняття управлінських рішень щодо асортиментної політики, регіональної експансії, ціноутворення та маркетингових стратегій.

\end{document}
