\documentclass{beamer}
\usepackage[utf8]{inputenc}
\usepackage[T2A]{fontenc}
\usepackage[ukrainian]{babel}
\usepackage{graphicx}
\usepackage{booktabs}
\usepackage{listings}
\usepackage{xcolor}

\usetheme{Madrid}
\usecolortheme{default}

\definecolor{techblue}{RGB}{30,144,255}
\definecolor{techdark}{RGB}{25,25,112}
\definecolor{techaccent}{RGB}{0,191,255}
\setbeamercolor{structure}{fg=techdark}
\setbeamercolor{frametitle}{bg=techblue,fg=white}
\setbeamercolor{block title}{bg=techblue,fg=white}
\setbeamercolor{item}{fg=techaccent}

\title{TechMarket: Проєктування та розробка високонавантаженої системи}
\subtitle{Фінальна презентація проєкту}
\author{Ярослав Кіщук}
\institute{Київський національний університет імені Тараса Шевченка \\ Факультет комп'ютерних наук та кібернетики}
\date{\today}

\begin{document}

\begin{frame}
    \titlepage
\end{frame}

\begin{frame}{Зміст}
    \tableofcontents
\end{frame}

\section{Вступ}
\begin{frame}{Про проєкт TechMarket}
    \textbf{Мета:} Розробка масштабованої системи електронної комерції для продажу техніки.
    
    \vspace{0.5cm}
    \textbf{Ключові компоненти:}
    \begin{itemize}
        \item Мікросервісна архітектура (Auth, Catalog, Orders, Payments).
        \item OLTP база даних (MySQL) для транзакційних операцій.
        \item ETL-пайплайн (Airflow) для обробки даних.
        \item OLAP сховище (PostgreSQL) для аналітики.
        \item Оркестрація контейнерів (Kubernetes).
    \end{itemize}
\end{frame}

\section{Архітектура системи}
\begin{frame}{Архітектура розгортання}
    \begin{figure}
        \centering
        \includegraphics[width=0.85\textwidth]{docs/images/TechMarket_Deployment.png}
        \caption{Детальна схема розгортання компонентів у Kubernetes}
        \label{fig:deployment}
    \end{figure}
\end{frame}

\section{Модель бази даних (OLTP)}
\begin{frame}[fragile]{OLTP Services: Auth \& Catalog}
    \begin{columns}[T]
        \begin{column}{0.48\textwidth}
            \begin{figure}
                \centering
                \includegraphics[width=\textwidth]{docs/images/Auth_ER.png}
                \caption{Auth Service}
                \label{fig:auth_er}
            \end{figure}
        \end{column}
        \begin{column}{0.48\textwidth}
            \begin{figure}
                \centering
                \includegraphics[width=\textwidth]{docs/images/Catalog_ER.png}
                \caption{Catalog Service}
                \label{fig:catalog_er}
            \end{figure}
        \end{column}
    \end{columns}
\end{frame}

\begin{frame}[fragile]{OLTP Services: Orders \& Payments}
    \begin{columns}[T]
        \begin{column}{0.48\textwidth}
            \begin{figure}
                \centering
                \includegraphics[width=\textwidth]{docs/images/Orders_ER.png}
                \caption{Orders Service}
                \label{fig:orders_er}
            \end{figure}
        \end{column}
        \begin{column}{0.48\textwidth}
            \begin{figure}
                \centering
                \includegraphics[width=\textwidth]{docs/images/Payments_ER.png}
                \caption{Payments Service}
                \label{fig:payments_er}
            \end{figure}
        \end{column}
    \end{columns}
\end{frame}

\section{ETL Процес}
\begin{frame}{ETL Пайплайн}
    \textbf{Інструменти:} Apache Airflow, Python (Pandas).
    
    \vspace{0.3cm}
    \textbf{Етапи:}
    \begin{enumerate}
        \item \textbf{Extract:} Вивантаження даних з MySQL (інкрементальне).
        \item \textbf{Transform:} Очищення, денормалізація, агрегація.
        \item \textbf{Load:} Завантаження в PostgreSQL (DWH).
    \end{enumerate}
    
    \begin{figure}
        \centering
        \includegraphics[width=0.5\textwidth]{docs/images/TechMarket_ETL_DWH.png}
        \caption{Схема ETL процесу}
        \label{fig:etl}
    \end{figure}
\end{frame}

\section{Сховище даних (DWH)}
\begin{frame}{Структура DWH (Схема "Зірка")}
    \begin{figure}
        \centering
        \includegraphics[width=0.7\textwidth]{docs/images/DWH_ER.png}
        \caption{Повна ER-діаграма сховища даних}
        \label{fig:dwh_er}
    \end{figure}
\end{frame}

\section{Лабораторні роботи}
\begin{frame}{Lab 1: Проєктування БД}
    \textbf{Завдання:} Розробка концептуальної та логічної моделі БД.
    
    \vspace{0.3cm}
    \textbf{Результат:}
    \begin{itemize}
        \item Створено ER-діаграми для 4 мікросервісів.
        \item Нормалізовано до 3НФ.
        \item Визначено зв'язки між сутностями.
    \end{itemize}
\end{frame}

\begin{frame}{Lab 2: Розгортання БД}
    \textbf{Завдання:} Розгортання OLTP БД у Docker.
    
    \vspace{0.3cm}
    \textbf{Результат:}
    \begin{itemize}
        \item Створено docker-compose конфігурацію.
        \item Ініціалізовано схеми для всіх сервісів.
        \item Згенеровано тестові дані (10K+ записів).
    \end{itemize}
\end{frame}

\begin{frame}{Lab 3: ETL Pipeline}
    \textbf{Завдання:} Побудова ETL процесу.
    
    \vspace{0.3cm}
    \textbf{Результат:}
    \begin{itemize}
        \item Реалізовано інкрементальне завантаження.
        \item Автоматизовано через Airflow DAG.
        \item Трансформація у схему "Зірка".
    \end{itemize}
\end{frame}

\begin{frame}{Lab 4: Розгортання у Kubernetes}
    \textbf{Завдання:} Оркестрація контейнерів.
    
    \vspace{0.3cm}
    \textbf{Результат:}
    \begin{itemize}
        \item Створено Kubernetes manifests.
        \item Налаштовано PersistentVolumes.
        \item Забезпечено автоматичний restart сервісів.
    \end{itemize}
\end{frame}

\begin{frame}{Lab 5: Аналітика та BI}
    \textbf{Завдання:} Побудова аналітичних запитів та візуалізація.
    
    \vspace{0.3cm}
    \textbf{Результат:}
    \begin{itemize}
        \item Створено SQL запити для аналізу продажів.
        \item Інтегровано Metabase для візуалізації.
        \item \textit{[Placeholder: Metabase Dashboard Screenshots]}
    \end{itemize}
\end{frame}

\begin{frame}{Lab 6: Оптимізація та масштабування}
    \textbf{Проблема:} Повільні запити при великих обсягах даних (1M+ рядків).
    
    \vspace{0.3cm}
    \textbf{Застосовані методи:}
    \begin{itemize}
        \item \textbf{Індексація:} B-Tree індекси для фільтрації та з'єднань.
        \item \textbf{Матеріалізовані представлення:} Кешування складних агрегацій.
        \item \textbf{VACUUM:} Очищення мертвих кортежів.
    \end{itemize}
    
    \vspace{0.3cm}
    \textbf{Результат:} Прискорення запитів у \textbf{308×} (з 450ms до 1.46ms).
\end{frame}

\section{Аналітичні результати}
\begin{frame}{Бізнес-аналітика}
    \textbf{Аналітичні можливості:}
    \begin{itemize}
        \item Аналіз трендів продажів.
        \item Сегментація клієнтів за активністю.
        \item Моніторинг інвентаря в реальному часі.
    \end{itemize}
    
    \vspace{0.3cm}
    \begin{figure}
        \centering
        \includegraphics[width=0.5\textwidth]{docs/images/TechMarket_Activity_BI_SalesTrend.png}
        \caption{Use Case діаграма BI системи}
        \label{fig:bi}
    \end{figure}
    
    \vspace{0.2cm}
    \textit{[Placeholder: Metabase аналітичні дашборди]}
\end{frame}

\section{Висновки}
\begin{frame}{Рефлексія}
    \textbf{Що вдалося:}
    \begin{itemize}
        \item Розробити реальну систему від проєктування до оптимізації.
        \item Здобути досвід роботи з сучасними технологіями (Kubernetes, Airflow).
        \item Навчитись налагоджувати та оптимізувати складні запити.
    \end{itemize}

    \vspace{0.3cm}
    \textbf{Найбільші виклики:}
    \begin{itemize}
        \item Налаштування Kubernetes — багато нюансів з volumes та networking.
        \item Оптимізація запитів на великих обсягах даних.
    \end{itemize}
    
    \vspace{0.3cm}
    \textbf{Висновок:}
    Проєкт дав можливість об'єднати теоретичні знання з практикою та побудувати реальну інфраструктуру даних.
\end{frame}

\begin{frame}
    \centering
    \Huge \textbf{Дякую за увагу!}
\end{frame}

\end{document}
