\documentclass[aspectratio=169]{beamer}

% ----------------------------
% Мова, кодування, шрифти
% ----------------------------
\usepackage[utf8]{inputenc}
\usepackage[T2A]{fontenc}
\usepackage[ukrainian,english]{babel}

% ----------------------------
% Пакети, які часто потрібні
% ----------------------------
\usepackage{amsmath, amssymb, amsfonts}
\usepackage{graphicx}
\usepackage{booktabs}
\usepackage{mathtools}
\usepackage{hyperref}

% ----------------------------
% Оформлення 
% ----------------------------
\usetheme{Madrid}          % або Warsaw, CambridgeUS, Boadilla, ...
\usecolortheme{seahorse}   % або default, dolphin, beaver, ...
\setbeamertemplate{navigation symbols}{} % без нижніх кнопок

% Нумерація слайдів
\setbeamertemplate{footline}[frame number]

% ----------------------------
% Метадані презентації
% ----------------------------
\title[Константа Кемені]{Константа Кемені і час до досягнення рівноваги}
\subtitle{За мотивами Peter Doyle (2005) та Bini-Hunter-Latouche-Meini-Taylor (2018)}
\author[Ярослав Кіщук]{Ярослав Кіщук}
\institute[КНУ]{КНУ, Факультет комп'ютерних наук та кібернетики}
\date{\today}

% ----------------------------
% Початок документа
% ----------------------------
\begin{document}

% Титульний слайд
\begin{frame}
    \titlepage
\end{frame}

% Слайд зі змістом
\begin{frame}{План доповіді}
    \tableofcontents
\end{frame}

% ----------------------------
% Розділ 0: Вступ
% ----------------------------
\section{Вступ}

\begin{frame}{Мотивація}
    \begin{itemize}
        \item Розглядаємо скінченний ергодичний марковський ланцюг з матрицею переходів \(P\) та стаціонарним розподілом \(\pi\).
        \medskip
        \item Константу Кемені \(K\) задається простою формулою як зважена сума середніх часів першого потрапляння, але має нетривіальну властивість: \textbf{вона не залежить від початкового стану}.
        \medskip
        \item У підручниках (напр., Grinstead-Snell) це подається як «загадка» або вправа з обіцянкою призу за інтуїтивне пояснення; стаття Doyle (2005) пропонує інтерпретацію як \emph{час до рівноваги}.
        \medskip
        \item Нова робота Bini-Hunter-Latouche-Meini-Taylor (2018) дає альтернативну, більш «фізичну» інтерпретацію через числа відвідувань та deviation matrix.
        \medskip
        \item \textbf{Мета доповіді:} побудувати єдину логічну картину навколо Константу Кемені, поєднавши:
        \begin{itemize}
            \item інтуїтивний підхід Doyle (очікуваний час до рівноваги),
            \item «counting visits» та deviation matrix у Bini et al.,
            \item стандартні матричні факти про марковські ланцюги.
        \end{itemize}
    \end{itemize}
\end{frame}

\begin{frame}{Постановка задачі}
    \begin{itemize}
        \item Нехай маємо скінченний ергодичний марковський ланцюг з:
        \begin{itemize}
            \item матрицею переходів \(P = (p_{ij})\),
            \item стаціонарним розподілом \(\pi = (\pi_1, \dots, \pi_n)\), \(\ \pi P = \pi\),
            \item часами першого потрапляння \(T_j\) та середніми значеннями \(m_{ij} = \mathbb{E}_i[T_j]\).
        \end{itemize}
        \medskip
        \item Визначаємо \textbf{Константу Кемені} як
        \[
            K \;=\; \sum_{j} \pi_j \, m_{ij},
        \]
        де індекс \(i\) — початковий стан ланцюга.
        \medskip
        \item Класичний факт: величина \(K\) \textbf{не залежить від вибору початкового стану} \(i\).
        \medskip
        \item \textbf{Основні питання доповіді:}
        \begin{itemize}
            \item Чому сума \(\sum_j \pi_j m_{ij}\) взагалі не залежить від \(i\)?
            \item Як інтерпретувати \(K\) як \emph{час досягнення рівноваги}?
            \item Як пов’язати \(K\) з deviation matrix та кількістю відвідувань станів?
            \item Як ці інтерпретації узагальнюються на безперервний час та ширші класи процесів?
        \end{itemize}
    \end{itemize}
\end{frame}

% ----------------------------
% Розділ 1: Базова теорія марковських ланцюгів
% ----------------------------
\section{Фон: скінченні ергодичні марковські ланцюги}

\begin{frame}{Марковський ланцюг і стаціонарний розподіл}
    \begin{itemize}
        \item Розглядаємо дискретний у часі марковський ланцюг \((X_n)_{n \ge 0}\) зі скінченним простором станів
        \[
            S = \{1,2,\dots,n\}.
        \]
        \medskip
        \item Його динаміка задається \textbf{матрицею переходів}
        \[
            P = (p_{ij})_{i,j \in S}, 
            \qquad 
            p_{ij} = \mathbb{P}(X_{n+1}=j \mid X_n = i),
        \]
        де кожний рядок є ймовірнісним: \(\sum_j p_{ij} = 1\).
        \medskip
        \item \textbf{Властивість Маркова:}
        \[
            \mathbb{P}(X_{n+1}=j \mid X_n=i, X_{n-1},\dots,X_0) 
            = \mathbb{P}(X_{n+1}=j \mid X_n=i) = p_{ij}.
        \]
        \medskip
        \item \textbf{Ергодичність} (у нашому контексті): ланцюг є
        \begin{itemize}
            \item \emph{незвідним} (irreducible): з будь-якого стану можна досягти будь-який інший,
            \item \emph{аперіодичним} (aperiodic): немає жорсткого циклічного періоду.
        \end{itemize}
        Це гарантує існування єдиного стаціонарного розподілу і збіжність до нього.
    \end{itemize}
\end{frame}

\begin{frame}{Стаціонарний розподіл та матриця рівноваги}
    \begin{itemize}
        \item \textbf{Стаціонарний розподіл} \(\pi = (\pi_1,\dots,\pi_n)\) — це ймовірнісний вектор, що задовольняє
        \[
            \pi P = \pi,
            \qquad
            \sum_{i=1}^n \pi_i = 1,
            \qquad
            \pi_i > 0.
        \]
        \medskip
        \item Для ергодичного ланцюга стаціонарний розподіл є \emph{єдиним}, і розподіл станів збігається до нього незалежно від старту:
        \[
            \lim_{n \to \infty} e_i^{\top} P^n = \pi
            \quad \text{для будь-якого початкового стану } i,
        \]
        де \(e_i\) — \(i\)-тий базисний вектор.
        \medskip
        \item Введемо \textbf{матрицю рівноваги}
        \[
            \Pi = \mathbf{1}\,\pi,
        \]
        де \(\mathbf{1} = (1,\dots,1)^{\top}\). Усі рядки \(\Pi\) однакові й дорівнюють \(\pi\).
        \medskip
        \item Для ергодичного ланцюга маємо 
        \(\lim_{n\to\infty} P^n = \Pi\), 
        тобто довгостроковий розподіл не залежить від початкового стану.
    \end{itemize}
\end{frame}

\begin{frame}{Матриці та часи першого потрапляння}
    \begin{itemize}
        \item Для стану \(j \in S\) \textbf{час першого потрапляння} (hitting time)
        \[
            T_j = \inf\{ n \ge 1 : X_n = j \}.
        \]
        \item \textbf{Середній час першого потрапляння} з \(i\) до \(j\):
        \[
            m_{ij} = \mathbb{E}_i[T_j], \qquad i \ne j,
        \]
        а за домовленістю \(m_{ii} = 0\).
        \medskip
        \item Збираємо їх у \textbf{матрицю середніх часів першого потрапляння}
        \[
            M = (m_{ij})_{i,j \in S}.
        \]
        \item Для кожного стану \(i\) розглянемо \textbf{час першого повернення}
        \[
            T_i^{+} = \inf\{ n \ge 1 : X_n = i \},
        \qquad
            r_i = \mathbb{E}_i[T_i^{+}] \ \text{(mean recurrence time)}.
        \]
        Класичний факт: для ергодичного ланцюга \(r_i = 1/\pi_i\).
        \medskip
        \item Введемо також діагональну матрицю
        \(R = \mathrm{diag}(r_1,\dots,r_n)\),
        яка просто містить середні часи повернення \(r_i\) на діагоналі.
    \end{itemize}
\end{frame}

% ----------------------------
% Розділ 2: Kemeny’s Constant
% ----------------------------
\section{Константа Кемені}

\begin{frame}{Визначення Константи Кемені}
    \begin{itemize}
        \item Нехай \(M = (m_{ij})\) — матриця середніх часів першого потрапляння:
        \[
            m_{ij} = \mathbb{E}_i[T_j], \quad i \neq j, 
            \qquad m_{ii} = 0.
        \]
        \medskip
        \item Нехай \(\pi = (\pi_1,\dots,\pi_n)\) — стаціонарний розподіл, \(\pi P = \pi\).
        \medskip
        \item Фіксуємо початковий стан \(i\) і визначаємо
        \[
            K_i \;=\; \sum_{j=1}^n \pi_j\, m_{ij}.
        \]
        \medskip
        \item \textbf{Твердження (Кемені):} величина \(K_i\) \emph{не залежить від вибору} \(i\).
        \medskip
        \item Отже, можна говорити про \textbf{Константи Кемені}
        \[
            K \;=\; \sum_{j} \pi_j\, m_{ij},
        \]
        маючи на увазі, що будь-який початковий стан дає одне й те саме значення.
    \end{itemize}
\end{frame}

\begin{frame}{Еквівалентна інтерпретація}
    \begin{itemize}
        \item Нехай випадковий «цільовий» стан \(J\) обирається незалежно від ланцюга за розподілом \(\pi\):
        \[
            \mathbb{P}(J = j) = \pi_j.
        \]
        \medskip
        \item Тоді час досягнення цієї випадкової цілі
        \[
            T_J = \text{час першого потрапляння в стан } J
        \]
        має сподівання
        \[
            \mathbb{E}_i[T_J] 
            \;=\; \sum_{j} \pi_j\, \mathbb{E}_i[T_j]
            \;=\; \sum_{j} \pi_j\, m_{ij}
            \;=\; K.
        \]
        \medskip
        \item \textbf{Отже:} \(K\) — це \emph{очікуваний час досягнення випадково обраного стану}, де «ціль» вибирається за стаціонарним розподілом.
        \medskip
        \item У подальшому ми інтерпретуватимемо це як \textbf{час досягнення рівноваги} (підхід Doyle) та пов’яжемо з deviation matrix (підхід Bini et al.).
    \end{itemize}
\end{frame}

\begin{frame}{Деякі базові властивості}
    \begin{itemize}
        \item Для скінченного ергодичного марковського ланцюга \(K\) завжди скінченна і додатна.
        \medskip
        \item \(K\) \textbf{інваріантна до перенумерації станів}:
        вона залежить лише від структури \(P\), а не від того, як ми позначаємо стани.
        \medskip
        \item Існує зв’язок із \textbf{фундаментальною матрицею}
        \[
            Z = (I - P + \Pi)^{-1},
        \]
        де \(I\) — одинична матриця, \(P\) — матриця переходів, \(\Pi = \mathbf{1}\,\pi\) — матриця стаціонарного розподілу.
        \medskip
        А саме (для нашої конвенції \(m_{ii}=0\)):
        \[
            K = \mathrm{tr}(Z) - 1.
        \]
        \item Також \(K\) має \textbf{спектральне представлення}:
        \[
            K = \sum_{k=2}^n \frac{1}{1 - \lambda_k},
        \]
        де \(\lambda_1 = 1 > \lambda_2 \ge \dots \ge \lambda_n\) — власні значення \(P\).
        \medskip
    \end{itemize}
\end{frame}

% ----------------------------
% Розділ 3: Підхід Дойла (time to equilibrium)
% ----------------------------
\section{Підхід Дойла: час до рівноваги}

\subsection{Інтерпретація}

\begin{frame}{Інтерпретація через час до рівноваги}
    \begin{itemize}
        \item Нагадаємо: 
        \[
            K_i = \sum_{j} \pi_j m_{ij},
        \quad
            m_{ij} = \mathbb{E}_i[T_j].
        \]
        \medskip
        \item Нехай випадковий «цільовий» стан \(J\) обирається за розподілом \(\pi\), незалежно від ланцюга:
        \[
            \mathbb{P}(J=j) = \pi_j.
        \]
        \item Тоді час досягнення цієї цілі
        \[
            T_J = T_J(i)
        \]
        має матсподівання
        \[
            \mathbb{E}_i[T_J] = \sum_j \pi_j \mathbb{E}_i[T_j] = \sum_j \pi_j m_{ij} = K_i.
        \]
        \medskip
        \item \textbf{Інтерпретація (Doyle):} \(K_i\) — це \emph{очікуваний час до «рівноваги»}, якщо рівновага моделюється випадковим станом, обраним за стаціонарним розподілом \(\pi\).
    \end{itemize}
\end{frame}

\subsection{Heuristic argument}

\begin{frame}{One-step аргумент та averaging property}
    \begin{itemize}
        \item Інтуїтивно: зробимо один крок зі стану \(i\), потрапляємо у стан \(k\) з імовірністю \(p_{ik}\).
        \medskip
        \item Здається природним написати
        \[
            K_i \stackrel{?}{=} 1 + \sum_k p_{ik} K_k,
        \]
        де «1» — це перший крок, а далі очікування з нового стану.
        \medskip
        \item Але є тонкий момент: ми могли \emph{вже} бути в рівновазі. Тоді робити ще один крок — «помилка». 
        \medskip
        \item Помилка трапляється з імовірністю \(\pi_i\), і вартість її в середньому дорівнює середньому часу повернення \(r_i = 1/\pi_i\), тобто очікуваний штраф:
        \[
            \pi_i \cdot r_i = \pi_i \cdot \frac{1}{\pi_i} = 1.
        \]
        \item Цей штраф «з’їдає» доданий нами \(+1\), і в результаті залишається
        \[
            K_i = \sum_k p_{ik} K_k.
        \]
    \end{itemize}
\end{frame}

\begin{frame}{Averaging property і принцип-максимуму}
    \begin{itemize}
        \item Рівність
        \[
            K_i = \sum_k p_{ik} K_k
        \]
        означає, що \(K_i\) є \textbf{середньозваженим} значенням \(K_k\) по сусідніх станах (з вагами \(p_{ik}\)).
        \medskip
        \item Такі функції мають \textbf{властивість усереднення} (averaging property) і задовольняють \textbf{maximum principle}:
        не можуть мати «внутрішнього» максимуму/мінімуму, якщо не є константою.
        \medskip
        \item Для незвідного ланцюга це означає, що всі значення \(K_i\) мусять збігатися:
        \[
            K_1 = K_2 = \dots = K_n = K.
        \]
        \item Отже, інтуїтивний one-step аргумент пояснює, чому \(\sum_j \pi_j m_{ij}\) не залежить від \(i\).
    \end{itemize}
\end{frame}

\subsection{Алгебраїчний доказ}

\begin{frame}{Фундаментальне рівняння для матриці M}
    \begin{itemize}
        \item Для \(i \neq j\): рекурсія для середніх часів першого потрапляння
        \[
            m_{ij} = 1 + \sum_k p_{ik} m_{kj}.
        \]
        \item У матричній формі (класичний результат Кемені–Снелла / Grinstead–Snell):
        \[
            (I - P) M = C - R,
        \]
        де
        \begin{itemize}
            \item \(M = (m_{ij})\) — матриця середніх часів першого потрапляння,
            \item \(C\) — матриця, всі елементи якої дорівнюють 1,
            \item \(R = \mathrm{diag}(r_1,\dots,r_n)\), де \(r_i = 1/\pi_i\) — середні часи повернення.
        \end{itemize}
        \item Doyle пропонує розглядати це рівняння як \textbf{фундаментальне} для аналізу константи Кемені.
    \end{itemize}
\end{frame}

\begin{frame}{Вивід константності K з фундаментального рівняння}
    \begin{itemize}
        \item Визначимо вектор
        \[
            k = (K_1,\dots,K_n)^{\top}, 
        \quad
            K_i = \sum_j \pi_j m_{ij}.
        \]
        Тоді у матричному вигляді
        \[
            k = M \pi^{\top}.
        \]
        \item Домножимо фундаментальне рівняння справа на \(\pi^{\top}\):
        \[
            (I - P) M \pi^{\top} = (C - R)\pi^{\top}.
        \]
        \item Маємо \(C\pi^{\top} = \mathbf{1}\) (бо \(\sum_j \pi_j = 1\)) і \(R\pi^{\top} = \mathbf{1}\) (бо \(r_i \pi_i = 1\)), отже
        \[
            (C - R)\pi^{\top} = 0.
        \]
        \item Тому
        \[
            (I - P)k = 0 
        \quad \Rightarrow \quad
            Pk = k.
        \]
        \item Вектор \(k\) є \textbf{фіксованим стовпчиком} для \(P\). Для ергодичного ланцюга такі стовпчики мусять бути константними, отже
        \[
            K_1 = \dots = K_n = K.
        \]
    \end{itemize}
\end{frame}

% ----------------------------
% Розділ 4: Підхід Bini-Hunter-Latouche-Meini-Taylor
% ----------------------------
\section{Підхід Bini-Hunter-Latouche-Meini-Taylor}

\subsection{Hitting times та відвідування станів}

\begin{frame}{Hitting times \texorpdfstring{$\theta_j$}{theta\_j}}
    \begin{itemize}
        \item У Bini et al. зручно працювати з \textbf{hitting times}
        \[
            \theta_j = \inf\{t \ge 0 : X_t = j\},
        \]
        тобто дозволяємо \(\theta_j = 0\), якщо \(X_0 = j\).
        \medskip
        \item Порівняння з нашими \(T_j\):
        \begin{itemize}
            \item якщо \(X_0 \ne j\), то \(\theta_j = T_j \ge 1\);
            \item якщо \(X_0 = j\), то \(\theta_j = 0 < T_j\).
        \end{itemize}
        \medskip
        \item Це дає альтернативну версію константи Кемені:
        \[
            K' \;=\; \sum_{j} \pi_j\, \mathbb{E}_i[\theta_j],
        \]
        де \(K' = K - 1\) (різниця лише в тому, як рахувати власний стан).
        \medskip
        \item Перевага такої форми: формули з \(\theta_j\) добре переносяться
        на безперервний час і природно пов’язуються з deviation matrix.
    \end{itemize}
\end{frame}

\begin{frame}{Підрахунок відвідувань \texorpdfstring{$N_j(n)$}{N\_j(n)}}
    \begin{itemize}
        \item Введемо кількість відвідувань стану \(j\) на проміжку \([0,n]\):
        \[
            N_j(n) = \sum_{t=0}^{n} \mathbf{1}\{X_t = j\}.
        \]
        \medskip
        \item Ключова лема (Bini et al.): для всіх \(i,j\)
        \[
            \frac{\mathbb{E}_i[\theta_j]}{\mathbb{E}_j[T_j]}
            \;=\;
            \lim_{n\to\infty}
            \big( \mathbb{E}_j[N_j(n)] - \mathbb{E}_i[N_j(n)] \big).
        \]
        \medskip
        \item Тобто:
        \begin{itemize}
            \item якщо стартуємо з \(j\), процес «відвідує \(j\) регулярно»;
            \item якщо стартуємо з \(i \ne j\), є \emph{початкова затримка} довжини \(\theta_j\);
            \item за цей час ми в середньому «втрачаємо»
            \(\mathbb{E}_i[\theta_j]/\mathbb{E}_j[T_j]\) відвідувань \(j\).
        \end{itemize}
        \medskip
        \item Просумувавши по \(j\) з вагами \(\pi_j\) і переставивши суму з границею,
        Bini et al. отримують формулу, з якої видно, що \(K'\) не залежить від \(i\).
    \end{itemize}
\end{frame}

\subsection{Deviation matrix}

\begin{frame}{Deviation matrix \texorpdfstring{$\mathcal{D}$}{D}}
    \begin{itemize}
        \item Для скінченного ергодичного ланцюга deviation matrix визначається як
        \[
            \mathcal{D}
            \;=\;
            \sum_{n \ge 0} \big( P^n - \Pi \big),
        \]
        де \(\Pi = \mathbf{1}\,\pi\) — матриця стаціонарного розподілу.
        \medskip
        \item Елемент \(\mathcal{D}_{ij}\) можна інтерпретувати як
        \[
            \mathcal{D}_{ij}
            =
            \lim_{n\to\infty}
            \left(
                \mathbb{E}_i[N_j(n)] - \mathbb{E}_\pi[N_j(n)]
            \right),
        \]
        тобто \emph{кумулятивне відхилення} числа відвідувань \(j\)
        при старті з \(i\) від числа відвідувань у рівновазі.
        \medskip
        \item Матрично \(\mathcal{D}\) задовольняє
        \[
            (I - P)\,\mathcal{D} = I - \Pi,
        \]
        і тісно пов’язана з груповою оберненою матрицею \((I - P)^\#\).
    \end{itemize}
\end{frame}

\begin{frame}{Вираз Константи Кемені через \texorpdfstring{$\mathcal{D}$}{D}}
    \begin{itemize}
        \item У Bini et al. показано, що діагональні елементи deviation matrix
        пов’язані з hitting times:
        \[
            \mathcal{D}_{jj} = \pi_j\, \mathbb{E}_\pi[\theta_j].
        \]
        \medskip
        \item Тоді
        \[
            \sum_{j} \mathcal{D}_{jj}
            \;=\;
            \sum_{j} \pi_j\, \mathbb{E}_\pi[\theta_j]
            \;=\;
            K',
        \]
        тобто
        \[
            K' = \mathrm{tr}(\mathcal{D}),
        \]
        а отже, з урахуванням \(K' = K - 1\),
        \[
            K = \mathrm{tr}(\mathcal{D}) + 1.
        \]
        \medskip
        \item Таким чином, Константа Кемені може бути інтерпретована як
        \textbf{слід deviation matrix} (з точною поправкою, що залежить від
        вибору між \(T_j\) і \(\theta_j\)).
        \medskip
        \item Це ще один спектрально-матричний опис \(K\), який добре працює
        і для безперервного часу, і (частково) для нескінченних просторів станів.
    \end{itemize}
\end{frame}

% ----------------------------
% Розділ 5: Узагальнення та застосування
% ----------------------------
\section{Узагальнення та застосування}

\begin{frame}{Зв'язок з mixing time та спектром}
    \begin{itemize}
        \item Константа Кемені відображає \textbf{типовий час, за який ланцюг
        «забуває» початковий стан} в середньому по цілях, зважених \(\pi\).
        \medskip
        \item Через спектральне представлення
        \[
            K = \sum_{k=2}^n \frac{1}{1 - \lambda_k}
        \]
        видно, що \(K\) чутлива до \textbf{наближеності власних значень} \(\lambda_k\)
        до 1:
        \begin{itemize}
            \item якщо \(|\lambda_2|\) близьке до 1, ланцюг змішується повільно, \(K\) велика;
            \item якщо всі \(\lambda_k\) далеко від 1, змішування швидке, \(K\) мала.
        \end{itemize}
        \medskip
        \item Це пов'язує Константу Кемені з \textbf{mixing time}:
        вона виступає більш «глобальним» спектральним показником швидкості змішування,
        на відміну від звичних оцінок лише через \(\lambda_2\).
        \medskip
        \item З точки зору застосувань:
        \begin{itemize}
            \item у випадкових блуканнях на графах \(K\) відображає «типову» довжину шляху до рівноваги;
            \item у марковських моделях (черги, надійність, Markov chain Monte Carlo)
            значення \(K\) дає орієнтир, скільки кроків потрібно, щоб система
            «загалом забула старт».
        \end{itemize}
    \end{itemize}
\end{frame}

\begin{frame}{Безперервний час і нескінченні простори}
    \begin{itemize}
        \item Ідеї, що стоять за Константою Кемені, переносяться і на
        \textbf{марковські процеси безперервного часу}, але про це у наступній серії.
    \end{itemize}
\end{frame}

\begin{frame}{Питання?}
    \centering
    {\LARGE Дякую за увагу!}\\[1em]

    \vspace{1em}

    \begin{flushleft}
    \footnotesize
    \textbf{Декілька фактів зі статті Doyle:}
    \begin{itemize}
        \item У задачі Grinstead--Snell про константу Кемені насправді було написано, що \emph{``a prize was offered''}, а не \emph{``a prize is offered''}~-- приз уже колись віддали.
        \item Laurie Snell надіслав Peter Doyle приз у \$50 поштою готівкою; перший лист так і не дійшов. Мораль: не надсилайте гроші готівкою поштою.
    \end{itemize}
    \end{flushleft}
\end{frame}

\end{document}