\documentclass[10pt]{beamer}

% Підтримка української мови
\usepackage{fontspec}
\usepackage{polyglossia}
\setmainlanguage{ukrainian}
\setotherlanguage{english}

% Налаштування шрифтів для кирилиці
\setmainfont{Times New Roman}
\setsansfont{Arial}
\setmonofont{Courier New}

% Математичні пакети
\usepackage{amsmath,amssymb,amsfonts}

\usepackage{graphicx}

% Графіка для космічного оформлення
\usepackage{tikz}
\usetikzlibrary{calc,decorations.pathmorphing,fadings}

% Космічна тема Beamer
\usetheme{default}
\useinnertheme{circles}
\useoutertheme{miniframes}

% Кольорова схема "космос"
\definecolor{spaceblack}{RGB}{10, 10, 25}
\definecolor{deepblue}{RGB}{20, 30, 80}
\definecolor{staryellow}{RGB}{255, 220, 100}
\definecolor{nebulacyan}{RGB}{80, 200, 255}
\definecolor{nebulapurple}{RGB}{150, 100, 200}

\setbeamercolor{background canvas}{bg=spaceblack}
\setbeamercolor{normal text}{fg=white}
\setbeamercolor{frametitle}{fg=nebulacyan, bg=deepblue!80!black}
\setbeamercolor{title}{fg=nebulacyan}
\setbeamercolor{subtitle}{fg=nebulapurple!70!white}
\setbeamercolor{author}{fg=staryellow}
\setbeamercolor{institute}{fg=white!80}
\setbeamercolor{date}{fg=white!60}
\setbeamercolor{item}{fg=nebulacyan}
\setbeamercolor{subitem}{fg=nebulapurple!70!white}
\setbeamercolor{block title}{fg=white, bg=deepblue}
\setbeamercolor{block body}{fg=white, bg=spaceblack!80!deepblue}
\setbeamercolor{structure}{fg=nebulacyan}
\setbeamercolor{footline}{fg=white!50, bg=spaceblack}
\setbeamercolor{headline}{fg=nebulacyan, bg=deepblue!50!black}
\setbeamercolor{navigation symbols}{fg=nebulacyan!50}

% Стиль заголовків слайдів
\setbeamerfont{frametitle}{size=\large, series=\bfseries}
\setbeamerfont{title}{size=\Large, series=\bfseries}
\setbeamerfont{subtitle}{size=\normalsize}

% Прибрати навігаційні символи
\setbeamertemplate{navigation symbols}{}

% Ввімкнути нумеровані підписи до рисунків і локалізувати мітку
\setbeamertemplate{caption}[numbered]
\renewcommand{\figurename}{Рис}

% Власний шаблон титульної сторінки
\defbeamertemplate*{title page}{customized}[1][]
{
  \begin{tikzpicture}[remember picture, overlay]
    % Градієнтний фон
    \fill[deepblue!30!spaceblack] (current page.south west) rectangle (current page.north east);
    % Зірки (випадкові точки)
    \foreach \i in {1,...,100} {
      \pgfmathsetmacro{\x}{rand*6}
      \pgfmathsetmacro{\y}{rand*4.5}
      \pgfmathsetmacro{\s}{0.3+rand*0.7}
      \fill[white, opacity=\s] (\x,\y) circle (0.02);
    }
    % Декоративне коло (планета/зоря)
    \shade[ball color=nebulacyan!60!deepblue, opacity=0.4] (-2.5,-1.5) circle (1.2);
    \shade[ball color=staryellow, opacity=0.8] (5.5,3.5) circle (0.6);
    % Орбіта
    \draw[nebulacyan!50, thick, dashed] (5.5,3.5) ellipse (2 and 0.8);
  \end{tikzpicture}
  
  \vspace{1cm}
  \begin{center}
    {\usebeamerfont{title}\usebeamercolor[fg]{title}\inserttitle}\\[4mm]
    {\usebeamerfont{subtitle}\usebeamercolor[fg]{subtitle}\insertsubtitle}\\[8mm]
    {\usebeamerfont{author}\usebeamercolor[fg]{author}\insertauthor}\\[3mm]
    {\usebeamerfont{institute}\usebeamercolor[fg]{institute}\insertinstitute}
  \end{center}
}

% -------------------------------
% Блок 0. Вступ і мотивація
% -------------------------------

\title[Орбітальна динаміка екзопланет]{Моделі руху матеріальної точки.\\[2mm] Орбітальна динаміка екзопланет і зона життя за моделлю Kopparapu}
\subtitle{Семінар з курсу «Моделювання динамічних систем»}
\author[Кіщук Я.Я.]{Кіщук Ярослав Ярославович}
\institute[КНУ]{Київський національний університет імені Тараса Шевченка \\ Факультет комп'ютерних наук та кібернетики}
\date{}

\begin{document}

% Слайд 1. Титульний
\begin{frame}
  \titlepage
\end{frame}

% Слайд 2. Мотивація: екзопланети й зона життя
\begin{frame}{Мотивація: екзопланети й зона життя}
  \begin{itemize}
    \item За останні десятиліття відкрито тисячі екзопланет (дані астрономічних оглядів, зокрема місій \textit{Kepler}, \textit{TESS} тощо).
    \item Ключове фізичне питання:
      \begin{itemize}
        \item які з цих планет можуть мати умови для існування рідкої води на поверхні?
      \end{itemize}
    \item Концепція \textbf{життєпридатної зони} (Habitable Zone, HZ):
      \begin{itemize}
        \item інтервал відстаней від зорі, де потік випромінювання дозволяє існування рідкої води,
        \item HZ залежить від світності $L$ та ефективної температури $T_{\text{eff}}$ зорі.
      \end{itemize}
    \item Для аналізу екзопланет потрібні:
      \begin{itemize}
        \item модель орбітальної динаміки (рух планети як матеріальної точки),
        \item модель енергетичного балансу, яка задає межі зони життя.
      \end{itemize}
  \end{itemize}
\end{frame}

% Слайд 3. Математична постановка й мета
\begin{frame}{Математична постановка й мета}
  \scriptsize
  \begin{itemize}
    \item \textbf{Модель руху:}
      \begin{itemize}
        \item планета моделюється як матеріальна точка маси $m$,
        \item що рухається в центральному ньютонівському полі тяжіння маси $M$ (зоря),
        \item двохтільна задача $\Rightarrow$ конічні орбіти (кола, еліпси, тощо).
      \end{itemize}


    \item \textbf{Орбітальні параметри:}
      \begin{itemize}
        \item велика піввісь $a$, ексцентриситет $e$, період $P$,
        \item ці параметри визначають траєкторію $r(t)$ та положення $(x(t),y(t))$ у часі.
      \end{itemize}


    \item \textbf{Модель зони життя Kopparapu:}
      \begin{itemize}
        \item за заданими $L$ та $T_{\text{eff}}$ зорі obчислюється інтервал $[d_{\text{in}}, d_{\text{out}}]$,
        \item тобто радіуси внутрішньої та зовнішньої меж життєпридатної зони.
      \end{itemize}


    \item \textbf{Мета доповіді:}
      \begin{itemize}
        \item побудувати строгий ланцюжок
        {\raggedright
        \[
          	ext{динаміка точки в полі } \frac{1}{r^2}
          \;\Longrightarrow\; \text{орбітальні елементи } (a,e,P)
          \;\Longrightarrow\; r(t)
          \;\Longrightarrow\; d_{\text{in}} \le r(t) \le d_{\text{out}},
        \]
        }
        \item і показати, як ці моделі використовуються для аналізу екзопланетних систем.
      \end{itemize}
  \end{itemize}
\end{frame}

% -------------------------------
% Блок 1. Модель руху матеріальної точки в центральному полі
% -------------------------------

\section{Модель руху матеріальної точки в центральному полі}

% Слайд 4. Двохтільна задача в ньютонівському формалізмі
\begin{frame}{Двохтільна задача в ньютонівському формалізмі}
  \begin{itemize}
    \item Розглядаємо систему:
      \begin{itemize}
        \item зоря маси $M$ (домінує гравітаційно),
        \item планета маси $m$ ($m \ll M$), яку моделюємо як матеріальну точку.
      \end{itemize}

    \medskip

    \item Закон всесвітнього тяжіння Ньютона:
      \[
        \mathbf{F}(\mathbf{r}) = -\,G\,\frac{M m}{r^{3}}\,\mathbf{r},
        \qquad r = \|\mathbf{r}\|.
      \]

    \medskip

    \item Другий закон Ньютона для планети:
      \[
        m \ddot{\mathbf{r}} = \mathbf{F}(\mathbf{r})
        \;\;\Longrightarrow\;\;
        \ddot{\mathbf{r}} = -\,G\,\frac{M}{r^{3}}\,\mathbf{r}.
      \]

    \medskip

    \item Властивості задачі:
      \begin{itemize}
        \item поле центральне $\Rightarrow$ момент імпульсу зберігається;
        \item траєкторія лежить в деякій фіксованій площині;
        \item система редукується до задачі про рух у площині.
      \end{itemize}
  \end{itemize}
\end{frame}

% Слайд 5. Перехід до полярних координат і інтеграли руху
\begin{frame}{Полярні координати та інтеграли руху}
  \begin{itemize}
    \item Вводимо полярні координати в площині орбіти:
      \[
        \mathbf{r}(t) = \bigl(r(t)\cos\theta(t),\, r(t)\sin\theta(t)\bigr).
      \]

    \medskip

    \item Рівняння руху в полярних координатах мають вигляд:
      \[
        \ddot r - r\,\dot\theta^{2} = -\,\frac{G M}{r^{2}},
        \qquad
        r^{2}\dot\theta = h = \text{const}.
      \]

    \medskip

    \item Інтеграл кутового моменту:
      \begin{itemize}
        \item $h = r^{2}\dot\theta$ --- кутовий момент на одиницю маси,
        \item $h = \text{const}$ $\Rightarrow$ друга формула Кеплера (рівність секторних площ за рівні проміжки часу).
      \end{itemize}

    \medskip

    \item Інтеграл енергії:
      \[
        E = \frac{m}{2}\bigl(\dot r^{2} + r^{2}\dot\theta^{2}\bigr)
            - \frac{G M m}{r}
            = \text{const}.
      \]
  \end{itemize}
\end{frame}

% Слайд 6. Лагранжевий формалізм
\begin{frame}{Лагранжевий формалізм для центрального поля}
  \begin{itemize}
    \item Кінетична та потенціальна енергії:
      \[
        T = \frac{m}{2}\bigl(\dot r^{2} + r^{2}\dot\theta^{2}\bigr),
        \qquad
        V(r) = -\,\frac{G M m}{r}.
      \]

    \medskip

    \item Лагранжіан системи:
      \[
        L(r,\theta,\dot r,\dot\theta)
        = T - V
        = \frac{m}{2}\bigl(\dot r^{2} + r^{2}\dot\theta^{2}\bigr)
          + \frac{G M m}{r}.
      \]

    \medskip

    \item Рівняння Лагранжа:
      \[
        \frac{d}{dt}\Bigl(\frac{\partial L}{\partial \dot q_i}\Bigr)
        - \frac{\partial L}{\partial q_i} = 0,
        \qquad q_i \in \{r,\theta\}.
      \]

    \medskip

    \item Для координати $\theta$:
      \[
        \frac{d}{dt}\bigl(m r^{2}\dot\theta\bigr) = 0
        \;\;\Longrightarrow\;\;
        m r^{2}\dot\theta = \text{const},
      \]
      що відтворює збереження кутового моменту.

    \item Для координати $r$ отримуємо те саме радіальне рівняння, що й з формалізму Ньютона.
  \end{itemize}
\end{frame}

% Слайд 7. Орбіти як конічні перерізи. Закони Кеплера
\begin{frame}{Орбіти як конічні перерізи та закони Кеплера}
  \begin{itemize}
    \item З інтегралів руху (енергія $E$, момент $h$) випливає,
      що траєкторія в центральному полі $\propto 1/r^{2}$
      є конічним перерізом.

    \medskip

    \item \textbf{Еліптична орбіта} (зв’язаний рух, $E<0$):
      \[
        r(\nu) = \frac{a\,(1 - e^{2})}{1 + e\cos\nu},
      \]
      де
      \begin{itemize}
        \item $a$ --- велика піввісь еліпса,
        \item $e$ --- ексцентриситет ($0 \le e < 1$),
        \item $\nu$ --- істинна аномалія (кут положення відносно перицентра).
      \end{itemize}

    \medskip

    \item \textbf{Третій закон Кеплера} (для $m \ll M$):
      \[
        P^{2} = \frac{4\pi^{2}}{G M}\,a^{3},
      \]
      де $P$ --- сидеричний період обертання.

    \medskip

    \item Узагальнення:
      \begin{itemize}
        \item $E = 0$ $\Rightarrow$ параболічна траєкторія ($e=1$),
        \item $E > 0$ $\Rightarrow$ гіперболічна траєкторія ($e>1$).
      \end{itemize}
  \end{itemize}
\end{frame}

% -------------------------------
% Блок 2. Від орбітальних елементів до положення планети
% -------------------------------

\section{Від орбітальних елементів до положення планети}

% Слайд 8. Орбітальні елементи як вхідні дані
\begin{frame}{Орбітальні елементи як вхідні дані}
  \begin{itemize}
    \item Для еліптичної орбіти планети навколо зорі (двохтільна задача) як вхідні дані зручно використовувати:
      \begin{itemize}
        \item велику піввісь $a$,
        \item ексцентриситет $e$,
        \item період обертання $P$,
        \item початкову фазу (напр., середню аномалію $M_0$ у момент $t=0$).
      \end{itemize}

    \medskip

    \item \textbf{Задача:}
      \begin{itemize}
        \item за заданим часом $t$ знайти \emph{поточне положення планети} на орбіті:
          \[
            r(t), \quad \theta(t) \quad \text{або} \quad (x(t),y(t)).
          \]
      \end{itemize}

    \medskip

    \item Стандартний ланцюжок перетворень:
      \[
        t \;\Longrightarrow\; M(t) \;\Longrightarrow\; E(t)
        \;\Longrightarrow\; \nu(t) \;\Longrightarrow\; r(t)
        \;\Longrightarrow\; (x(t),y(t)).
      \]
  \end{itemize}
\end{frame}

% Слайд 9. Середня аномалія M(t)
\begin{frame}{Середня аномалія $M(t)$}
  \begin{itemize}
    \item \textbf{Середня аномалія} $M$ визначається як кут,
      який зростає рівномірно з часом:
      \[
        M(t) = n\,t + M_0,
        \qquad n = \frac{2\pi}{P}.
      \]

    \medskip

    \item $n$ --- \textbf{середній рух} (середня кутова швидкість),
      $M_0$ --- значення $M$ при $t=0$.

    \medskip

    \item Геометричний зміст:
      \begin{itemize}
        \item $M$ --- кутова координата \emph{уявної} точки,
          що рухається \textbf{рівномірно по колу}
          радіуса $a$ з періодом $P$.
        \item Для реальної планети рух по еліпсу нерівномірний,
          але $M$ використовується як параметр часу.
      \end{itemize}

    \medskip

    \item Далі: $M(t)$ входить до \textbf{рівняння Кеплера}, яке пов’язує $M$ з ексцентричною аномалією~$E$.
  \end{itemize}
\end{frame}

% Слайд 10. Рівняння Кеплера і ексцентрична аномалія E
\begin{frame}{Рівняння Кеплера і ексцентрична аномалія $E$}
  \begin{itemize}
    \item \textbf{Ексцентрична аномалія} $E$ --- допоміжний кут,
      який параметризує положення точки на еліпсі.

    \medskip

    \item Для еліптичної орбіти ($0 \le e < 1$) виконується
      \textbf{рівняння Кеплера}:
      \[
        M = E - e \sin E.
      \]

    \medskip

    \item Це трансцендентне рівняння:
      \begin{itemize}
        \item немає виразу $E(M)$ в елементарних функціях;
        \item для заданих $M$ і $e$ потрібно чисельно знайти корінь
          $f(E) = E - e\sin E - M = 0$.
      \end{itemize}

    \medskip

    \item \textbf{Чисельне розв’язання:}
      \begin{itemize}
        \item на практиці часто використовують ітераційні методи,
          зокрема метод Ньютона–Рафсона
          для рівняння $f(E) = 0$;
        \item за малих $e$ як початкове наближення беруть $E_0 \approx M$,
          ітерації швидко збігаються.
      \end{itemize}
  \end{itemize}
\end{frame}

% Слайд 11. Перехід від E до істинної аномалії ν та радіуса r
\begin{frame}{Перехід від $E$ до істинної аномалії $\nu$ та радіуса $r$}
  \begin{itemize}
    \item \textbf{Істинна аномалія} $\nu$ --- фактичний кут
      положення планети відносно перицентра орбіти.

    \medskip

    \item Знайшовши $E$, обчислюємо $\nu$ за класичною формулою:
      \[
        \tan\frac{\nu}{2}
        = \sqrt{\frac{1+e}{1-e}}\,
          \tan\frac{E}{2}.
      \]

    \medskip

    \item Потім обчислюємо радіус-вектор:
      \[
        r = \frac{a(1 - e^{2})}{1 + e\cos\nu}
        \quad\text{(еквівалентно,}\;\;
        r = a(1 - e\cos E)\text{)}.
      \]

    \medskip

    \item Таким чином, для кожного моменту часу $t$ маємо:
      \[
        t \Rightarrow M(t) \Rightarrow E(t)
        \Rightarrow \nu(t) \Rightarrow r(t).
      \]
  \end{itemize}
\end{frame}

% Слайд 12. Координати (x, y) в площині орбіти
\begin{frame}{Координати $(x,y)$ в площині орбіти}
  \begin{itemize}
    \item Оберемо систему координат так, щоб:
      \begin{itemize}
        \item зоря маси $M$ знаходилася в початку координат $(0,0)$,
        \item перицентр орбіти лежав на додатній осі $Ox$.
      \end{itemize}

    \medskip

    \item Тоді полярні координати $(r(t),\nu(t))$
      переходять у декартові за стандартними формулами:
      \[
        x(t) = r(t)\cos\nu(t), \qquad
        y(t) = r(t)\sin\nu(t).
      \]

    \medskip

    \item \textbf{Результат:}
      \begin{itemize}
        \item для кожного часу $t$ ми маємо однозначно визначене
          положення планети $(x(t),y(t))$ на площині орбіти;
        \item далі ці координати можуть бути використані для
          візуалізації орбіти та аналізу того,
          чи потрапляє планета в життєпридатну зону.
      \end{itemize}

    \medskip

    \item (За потреби можна додати повороти/нахил орбіти
      для переходу до 3D, але базова 2D-модель вже повністю визначена.)
  \end{itemize}
\end{frame}

% -------------------------------
% Блок 3. Модель життєпридатної зони Kopparapu
% -------------------------------

\section{Модель життєпридатної зони Kopparapu}

% Слайд 13. Поняття життєпридатної зони
\begin{frame}{Поняття життєпридатної зони}
  \begin{itemize}
    \item \textbf{Життєпридатна зона} (Habitable Zone, HZ) --- 
    інтервал відстаней від зорі, в якому на поверхні планети земного типу
    \emph{може} існувати рідка вода.
    
    \medskip

    \item На рівні енергетики:
      \begin{itemize}
        \item потік зіркового випромінювання на орбіті планети
              має лежати в деякому діапазоні,
        \item занадто великий потік $\Rightarrow$ вода випаровується (перегрів, «парникова Венера»),
        \item занадто малий потік $\Rightarrow$ вода замерзає (глобальне заледеніння).
      \end{itemize}

    \medskip

    \item Математично HZ задається інтервалом радіусів:
      \[
        [d_{\text{in}}, d_{\text{out}}]
      \]
      навколо зорі, де $d_{\text{in}}$ --- внутрішня межа HZ,
      $d_{\text{out}}$ --- зовнішня межа.

    \medskip

    \item Положення HZ залежить від:
      \begin{itemize}
        \item світності зорі $L$,
        \item ефективної температури $T_{\text{eff}}$,
        \item (у більш точній моделі) від маси та атмосфери планети.
      \end{itemize}
  \end{itemize}
\end{frame}

% Слайд 14. Ефективний потік S_{\text{eff}}(T_*)
\begin{frame}{Ефективний потік $S_{\text{eff}}(T_*)$}
  \begin{itemize}
    \item Kopparapu et al. вводять \textbf{ефективний потік} $S_{\text{eff}}$:
      \begin{itemize}
        \item безрозмірна величина,
        \item $S_{\text{eff}} = 1$ відповідає середньому потоку на орбіті Землі.
      \end{itemize}

    \medskip

    \item Для кожної межі HZ ($\text{inner}$, $\text{outer}$)
          $S_{\text{eff}}$ задається як поліном від
          \[
            T^{*} = T_{\text{eff}} - 5780~\text{K}.
          \]

    \medskip

    \item \textbf{Поліноміальна апроксимація Kopparapu:}
      \[
        S_{\text{eff}}(T_*)
        = S_{\text{eff},\odot}
          + a T^{*}
          + b (T^{*})^{2}
          + c (T^{*})^{3}
          + d (T^{*})^{4},
      \]
      де
      \begin{itemize}
        \item $S_{\text{eff},\odot}$, $a$, $b$, $c$, $d$ --- табличні коефіцієнти,
        \item різні набори коефіцієнтів для:
          \begin{itemize}
            \item внутрішньої межі (наприклад, «runaway greenhouse»),
            \item зовнішньої межі (наприклад, «maximum greenhouse»),
            \item різних мас планети (0.1, 1, 5~$M_\oplus$).
          \end{itemize}
      \end{itemize}

    \medskip

    \item Таким чином, $S_{\text{eff}}$ --- функція температури зорі $T_{\text{eff}}$,
          яка враховує спектральний розподіл випромінювання.
  \end{itemize}
\end{frame}

% Слайд 15. Від S_{\text{eff}} до відстані d
\begin{frame}{Перехід від $S_{\text{eff}}$ до відстані $d$}
  \footnotesize
  \begin{itemize}
    \item Потік випромінювання від зорі світності $L$ на відстані $d$:
      \[
        F(d) = \frac{L}{4\pi d^{2}}.
      \]

    \item Нормування до сонячного випадку (Земля):
      \[
        S_{\text{eff}}
        = \frac{F(d)}{F_\oplus}
        = \frac{L/L_\odot}{d^{2}},
      \]
      якщо $d$ вимірюємо в астрономічних одиницях.

    \item Формула для відстані:
      \[
        d = \sqrt{\frac{L/L_\odot}{S_{\text{eff}}}}.
      \]

    \item Межі HZ:
      \[
        d_{\text{in}} = \sqrt{\frac{L/L_\odot}{S_{\text{eff}}^{(\text{inner})}(T_*)}},\quad
        d_{\text{out}} = \sqrt{\frac{L/L_\odot}{S_{\text{eff}}^{(\text{outer})}(T_*)}}.
      \]

    \item Таким чином, для заданих $L$ і $T_{\text{eff}}$ ми аналітично
          отримуємо радіуси меж життєпридатної зони.
  \end{itemize}
\end{frame}

% Слайд 16. Фізичний зміст внутрішньої та зовнішньої межі
\begin{frame}{Фізичний зміст $d_{\text{in}}$ та $d_{\text{out}}$}
  \begin{itemize}
    \item \textbf{Внутрішня межа} $d_{\text{in}}$ (runaway greenhouse):
      \begin{itemize}
        \item за $d < d_{\text{in}}$ потік занадто великий,
        \item температура поверхні зростає,
              океани випаровуються,
              парниковий ефект самопідсилюється,
        \item врешті-решт вода повністю переходить у газову фазу
              та може бути втрачена.
      \end{itemize}

    \medskip

    \item \textbf{Зовнішня межа} $d_{\text{out}}$ (maximum greenhouse):
      \begin{itemize}
        \item за $d > d_{\text{out}}$ потік занадто малий,
        \item навіть максимальний парниковий ефект (насичена CO$_2$ атмосфера)
              не здатен утримати температуру вище 0°C,
        \item планета переходить у стан глобального заледеніння.
      \end{itemize}

    \medskip

    \item \textbf{Підсумок:}
      \begin{itemize}
        \item інтервал $[d_{\text{in}}, d_{\text{out}}]$ задає
              необхідну (але не достатню) умову для існування рідкої води;
        \item надалі ми будемо порівнювати орбіту $r(t)$ планети
              з цим інтервалом.
      \end{itemize}
  \end{itemize}
\end{frame}

% -------------------------------
% Блок 4. Динамічна система «орбіта + зона життя»
% -------------------------------

\section{Динамічна система «орбіта + зона життя»}

% Слайд 17. Орбіта r(t) та інтервал [d_in, d_out]
\begin{frame}{Орбіта $r(t)$ та інтервал $[d_{\text{in}}, d_{\text{out}}]$}
  \begin{itemize}
    \item Для еліптичної орбіти маємо
      \[
        r(t) = \frac{a(1 - e^{2})}{1 + e\cos\nu(t)},
      \]
      де $\nu(t)$ --- істинна аномалія, знайдена через
      $M(t)$ та $E(t)$.

    \medskip

    \item Життєпридатна зона зорі задається інтервалом радіусів:
      \[
        [d_{\text{in}}, d_{\text{out}}],
      \]
      де $d_{\text{in}}$, $d_{\text{out}}$ обчислені за моделлю Kopparapu.

    \medskip

    \item \textbf{Умова перебування планети в HZ} в момент часу $t$:
      \[
        d_{\text{in}} \;\le\; r(t) \;\le\; d_{\text{out}}.
      \]

    \medskip

    \item Таким чином, динамічна система
      \[
        t \mapsto r(t)
      \]
      поєднується з \emph{статично} заданим інтервалом $[d_{\text{in}}, d_{\text{out}}]$,
      утворюючи задачу про час перебування траєкторії всередині заданого кільця.
  \end{itemize}
\end{frame}

% Слайд 18. Частка періоду в життєпридатній зоні
\begin{frame}{Частка періоду в життєпридатній зоні}
  \begin{itemize}
    \item Нехай $P$ --- період обертання планети по орбіті.

    \medskip

    \item Введемо \textbf{індикаторну функцію} перебування в HZ:
      \[
        \chi_{\text{HZ}}(t)
        =
        \begin{cases}
          1, & d_{\text{in}} \le r(t) \le d_{\text{out}},\\[1mm]
          0, & \text{інакше}.
        \end{cases}
      \]

    \medskip

    \item \textbf{Частка періоду}, яку планета проводить у HZ:
      \[
        \tau_{\text{HZ}}
        = \frac{1}{P} \int_{0}^{P} \chi_{\text{HZ}}(t)\,dt,
        \qquad 0 \le \tau_{\text{HZ}} \le 1.
      \]

    \medskip

    \item Для майже кругової орбіти ($e \approx 0$):
      \begin{itemize}
        \item якщо $a \in [d_{\text{in}}, d_{\text{out}}]$, то $\tau_{\text{HZ}} \approx 1$;
        \item якщо $a$ поза цим інтервалом, то $\tau_{\text{HZ}} \approx 0$.
      \end{itemize}

    \medskip

    \item Для ексцентричної орбіти ($e > 0$):
      \begin{itemize}
        \item можливі проміжні значення $0 < \tau_{\text{HZ}} < 1$;
        \item планета частину орбіти проводить в HZ, частину --- поза нею.
      \end{itemize}
  \end{itemize}
\end{frame}

% Слайд 19. Обчислення τ_HZ через аномалію ν
\begin{frame}{Обчислення $\tau_{\text{HZ}}$ через істинну аномалію $\nu$}
  \footnotesize
  \begin{itemize}
    \item Використаємо зв'язок між часом і кутом $\theta$ (або $\nu$)
      з другого закону Кеплера:
      \[
        r^{2} \dot\theta = h = \text{const}
        \;\;\Longrightarrow\;\;
        \frac{d t}{d\theta} = \frac{r^{2}(\theta)}{h}.
      \]

    \medskip

    \item Якщо ототожнити $\theta \equiv \nu$,
      тоді для одного періоду:
      \[
        P = \int_{0}^{2\pi} \frac{r^{2}(\nu)}{h}\,d\nu.
      \]

    \medskip

    \item Час, проведений у HZ:
      \[
        T_{\text{HZ}}
        = \int_{\{\nu: d_{\text{in}} \le r(\nu) \le d_{\text{out}}\}}
          \frac{r^{2}(\nu)}{h}\,d\nu.
      \]

    \medskip

    \item Звідси
      \[
        \tau_{\text{HZ}} = \frac{T_{\text{HZ}}}{P}
        = \frac{\displaystyle
              \int_{\{\nu: d_{\text{in}} \le r(\nu) \le d_{\text{out}}\}}
                  r^{2}(\nu)\,d\nu}
               {\displaystyle
              \int_{0}^{2\pi} r^{2}(\nu)\,d\nu}.
      \]

    \medskip

    \item Це дає \emph{строге інтегральне визначення} $\tau_{\text{HZ}}$
          через відому функцію $r(\nu)$, де
          \(
            r(\nu) = \dfrac{a(1-e^{2})}{1+e\cos\nu}.
          \)
  \end{itemize}
\end{frame}

% Слайд 20. Вплив параметрів e, L, T_eff
\begin{frame}{Вплив параметрів $e$, $L$, $T_{\text{eff}}$}
  \begin{itemize}
    \item \textbf{Ексцентриситет $e$:}
      \begin{itemize}
        \item при $e = 0$ радіус $r(t) \equiv a$ сталий;
        \item при збільшенні $e$:
          \begin{itemize}
            \item $r_{\min} = a(1-e)$ (перицентр),
            \item $r_{\max} = a(1+e)$ (апоцентр),
          \end{itemize}
        \item чим більший $e$, тим ширший діапазон $r(t)$
              і тим складніша структура входів/виходів у HZ.
      \end{itemize}

    \medskip

    \item \textbf{Світність $L$:}
      \begin{itemize}
        \item межі HZ масштабуються як $d \propto \sqrt{L}$,
        \item для більш яскравої зорі HZ зміщується на більші радіуси,
        \item для слабшої зорі --- навпаки, ближче до зорі.
      \end{itemize}

    \medskip

    \item \textbf{Ефективна температура $T_{\text{eff}}$:}
      \begin{itemize}
        \item впливає на $S_{\text{eff}}(T_*)$ в моделі Kopparapu,
        \item змінює положення меж HZ навіть при фіксованому $L$,
        \item пов’язане зі спектральним типом зорі та спектральним розподілом випромінювання.
      \end{itemize}
  \end{itemize}
\end{frame}

% -------------------------------
% Блок 5. Приклади та ілюстрації
% -------------------------------

\section{Приклади та ілюстрації}

% Слайд 21. Приклад 1: майже кругова орбіта всередині HZ
\begin{frame}{Приклад 1: майже кругова орбіта всередині HZ}
  \begin{columns}
    \begin{column}{0.55\textwidth}
      \begin{itemize}
        \item Орбіта з малим ексцентриситетом $e \approx 0$.
        \item Велика піввісь $a$ вибрана так, що
              \[
                d_{\text{in}} < a < d_{\text{out}}.
              \]
        \item Тоді $r(t) \approx a$ для всіх $t$,
              і умова перебування в HZ виконується протягом усього періоду.
        \item \textbf{Частка періоду:} $\tau_{\text{HZ}} \approx 1$.
      \end{itemize}
    \end{column}
    \begin{column}{0.45\textwidth}
      \begin{figure}
        \centering
        % Заміни file_name.pdf / .png на свою ілюстрацію
        \includegraphics[width=\linewidth]{circular_orbit_hz.pdf}
        \caption{Схематична орбіта всередині HZ.}
      \end{figure}
    \end{column}
  \end{columns}
\end{frame}

% Слайд 22. Приклад 2: ексцентрична орбіта, що перетинає HZ
\begin{frame}{Приклад 2: ексцентрична орбіта, що перетинає HZ}
  \begin{columns}
    \begin{column}{0.55\textwidth}
      \begin{itemize}
        \item Орбіта з помітним ексцентриситетом $e > 0$.
        \item Перицентр $r_{\min} < d_{\text{in}}$:
          \begin{itemize}
            \item поблизу перицентра планета перебуває \emph{всередині}
                  внутрішньої межі (надто жарко).
          \end{itemize}
        \item Апоцентр $r_{\max} > d_{\text{out}}$:
          \begin{itemize}
            \item поблизу апоцентра планета \emph{поза} HZ
                  (надто холодно).
          \end{itemize}
        \item Лише на частині орбіти виконується
              \(
                d_{\text{in}} \le r(t) \le d_{\text{out}}.
              \)
        \item \textbf{Частка періоду:} $0 < \tau_{\text{HZ}} < 1$.
      \end{itemize}
    \end{column}
    \begin{column}{0.45\textwidth}
      \begin{figure}
        \centering
        % Заміни file_name.pdf / .png на свою ілюстрацію
        \includegraphics[width=\linewidth]{eccentric_orbit_hz.pdf}
        \caption{Ексцентрична орбіта, що перетинає HZ.}
      \end{figure}
    \end{column}
  \end{columns}
\end{frame}

% Слайд 23. Коментарі до прикладів
\begin{frame}{Коментарі до прикладів}
  \begin{itemize}
    \item Обидва приклади будуються на одних і тих самих формулах:
      \[
        r(\nu) = \frac{a(1-e^{2})}{1 + e\cos\nu},
        \qquad
        d_{\text{in}}, d_{\text{out}}\ \text{з моделі Kopparapu}.
      \]

    \medskip

    \item Різниця полягає в:
      \begin{itemize}
        \item значенні ексцентриситету $e$,
        \item співвідношенні $a$ з $d_{\text{in}}$ та $d_{\text{out}}$,
        \item а також у відповідних значеннях $\tau_{\text{HZ}}$.
      \end{itemize}

    \medskip

    \item На практиці:
      \begin{itemize}
        \item з орбітальних елементів (виміряних астрономічно)
              та параметрів зорі (світність, $T_{\text{eff}}$)
              можна \emph{обчислити}, як багато часу
              планета проводить у життєпридатній зоні.
      \end{itemize}
  \end{itemize}
\end{frame}

% -------------------------------
% Блок 6. Висновки та література
% -------------------------------

\section{Висновки та література}

% Слайд 24. Підсумки (компактна версія)
\begin{frame}{Підсумки}
  \small
  \begin{itemize}
    \item \textbf{Динамічна модель}
      \begin{itemize}
        \item Рух планети описано як рух матеріальної точки в центральному полі $1/r^{2}$.
        \item Отримано конічні орбіти (еліпси) та закони Кеплера.
      \end{itemize}

    \medskip

    \item \textbf{Від орбітальних елементів до координат}
      \begin{itemize}
        \item Орбіта задається параметрами $(a,e,P,M_0)$.
        \item Побудовано ланцюжок:
        \[
          t \Rightarrow M(t) \Rightarrow E(t)
          \Rightarrow \nu(t) \Rightarrow r(t) \Rightarrow (x(t),y(t)).
        \]
        \item Рівняння Кеплера $M = E - e\sin E$ розв’язується чисельно.
      \end{itemize}

    \medskip

    \item \textbf{Зона життя Kopparapu}
      \begin{itemize}
        \item За $L$ та $T_{\text{eff}}$ зорі обчислено
              ефективний потік $S_{\text{eff}}(T_*)$
              і межі $d_{\text{in}}, d_{\text{out}}$.
        \item Критерій перебування планети в HZ:
              \(
                d_{\text{in}} \le r(t) \le d_{\text{out}}.
              \)
      \end{itemize}

    \medskip

    \item \textbf{Головний висновок}
      \begin{itemize}
        \item Поєднання орбітальної динаміки з моделлю HZ
              дозволяє кількісно оцінити частку періоду
              $\tau_{\text{HZ}}$, яку екзопланета проводить
              у життєпридатній зоні своєї зорі.
      \end{itemize}
  \end{itemize}
\end{frame}


% Слайд 25. Джерела та література
\begin{frame}{Джерела та література}
  \small
  \begin{itemize}
    \item Ландау Л.Д., Лифшиц Е.М. \textit{Механика}. —
          М.: Наука, 1988. — (Теоретическая физика, т.~1).

    \item Гантмахер Ф.Р. \textit{Лекции по аналитической механике}. —
          М.: Наука, 1966.

    \item Kasting J.F., Whitmire D.P., Reynolds R.T.\\
          \textit{Habitable Zones around Main Sequence Stars}. —
          Icarus, 101, 108–128 (1993).

    \item Kopparapu R.K., Ramirez R., Kasting J.F. et al.\\
          \textit{Habitable Zones around Main-Sequence Stars: New Estimates}. —
          Astrophysical Journal, 765, 131 (2013).

    \item Kopparapu R.K., Ramirez R.M., SchottelKotte J. et al.\\
          \textit{Habitable Zones around Main-Sequence Stars: Dependence on Planetary Mass}. —
          Astrophysical Journal Letters, 787, L29 (2014).

    \item Онлайн-ресурси:
      \begin{itemize}
        \item https://www.exoplanetvisualizer.com/,
      \end{itemize}
  \end{itemize}
\end{frame}


\end{document}